% !TEX root = ../main.tex
\section{AreaComp}

\subsection{Prinzip}

\begin{frame}
	\frametitle{Algorithmus}

	\note<1->[item]{Kommen nun zum Algorithmus}
	\note<1->[item]{Eingabe/Ausgabe: \begin{itemize}
			\item Eingabe: zu komprimierender String
			\item Ausgabe: Straight-Line-Grammatik für $s$
		\end{itemize}
	}
	\note<1->[item]{Berechne Enhanced Suffix Array}
	\note<1->[item]{
		Generieren der Menge von LCP-Intervallen \begin{itemize}
			\item Intervalle des Kind-Intervallbaums
		\end{itemize}
	}
	\note<1->[item]{
		Bis alle Intervalle bearbeitet wurden: (solange $Q$ nicht leer) \begin{itemize}
			\item Als am nützlichsten eingeschätztes Intervall entnehmen
			\item Produktionsregel erzeugen, die den Substring erzeugt, der vom LCP-Intervall beschrieben wird
			\item Wo möglich, den Substring durch das Nichtterminal der Produktionsregel ersetzen
		\end{itemize}
	}

	\note<1->[item]{
		Falls keine Intervalle mehr übrig, terminiere
	}


	\begin{algorithm}[H]
		\KwIn{$s:$ Eingabestring}
		\KwOut{$G:$ Grammatik mit $L(G) = \{s\}$}
		Berechne Enhanced Suffix Array für $s\$$\;
		Generiere Menge $Q$ von LCP-Intervallen\;
		%Ordne mithilfe der Flächenfunktion in Prioritätswarteschlange $Q$\;
		\While{$Q \neq \emptyset$}{
			Entnimm nützlichstes LCP-Intervall\;
			Generiere neue Produktionsregel für den Substring\;
			Ersetze den Substring, wo möglich\;
		}
		Gib die Grammatik zurück\;
		\caption{AreaComp}
	\end{algorithm}
\end{frame}

\begin{frame}
	\frametitle{Offene Fragen}

	\note<1->[item]{Priorisierung: Am naheliegendsten ist Prioritätswarteschlange}
	\note<2->[item]{Offene Frage: Was heißt \enquote{Nützlichkeit} und wie bestimmen wir diese?}
	\note<3->[item]{Wir ersetzen Teile des Strings durch Produktionsregeln und Nichtterminale \begin{itemize}
			\item Wie speichern wir \enquote{Ersetzungen}
		\end{itemize}
	}

	\begin{itemize}
		\item<1-> Priorisierung mithilfe von Prioritätswarteschlange
		\item<2-> Wie bestimmen wir \enquote{Nützlichkeit}?
		\item<3-> Brauchen Möglichkeit \enquote{Ersetzungen} zu speichern
	\end{itemize}
\end{frame}

\begin{frame}
	\frametitle{Flächenfunktionen}

	\note[item]<1->{Zur Priorisierung von LCP-Intervallen}
	\note[item]<1->{Ein Kernstück des Algorithmus}
	\note[item]<2->{Flächenfunktionen}
	\note[item]<2->{Zunächst Hilfsfunktionen definieren: \begin{itemize}
		\item Werden zur Berechnung der Flächenfunktion benutzt 
		\item Höhe: Minimaler Wert in dem LCP-Intervall. Ist die Länge des beschriebenen Substrings
		\item Breite: Anzahl Vorkommen des Substrings, der durch das LCP-Intervall beschrieben wird
	\end{itemize}}
	\note[item]<3->{Flächenfunktion \begin{itemize}
		\item Im Grunde einfache Definition
		\item Nimmt die Intervallgrenzen und gibt eine natürliche Zahl zurück
		\item Nutzt Höhe und Breite zum Bewerten des Intervalls
		\item Idee ist, dass die Funktion bei nützlich eingeschätzten Intervallen hohe Zahl liefert
	\end{itemize}}

	\begin{block}<2->{Breite/Höhe eines LCP-Intervalls}
		Sei $LCP[i..j]$ ein Intervall im LCP-Array mit $i \leq j$
		\begin{itemize}
			\item Höhe: $H(i,j) = \min \{LCP[k]\ |\ i \leq k \leq j\}$
			\item Breite: $W(i,j) = j - i + 2$
		\end{itemize} 
	\end{block}

	\begin{block}<3->{Flächenfunktion}
		Eine Funktion $A: \mathbb{N}_0 \times \mathbb{N}_0 \rightarrow \mathbb{N}_0$. 
		Die Argumente sind die Intervallgrenzen eines LCP-Intervalls. 
		Nutzt die Höhe und Breite des LCP-Intervalls um das Intervall zu bewerten.
	\end{block}

\end{frame}

\begin{frame}
	\frametitle{Beispiel}

	\note[item]{Ein Beispiel}
	\note[item]{Komprimierung von dem abgebildeten String}
	\note[item]{Flächenfunktion: Multiplikation von Höhe und Breite \begin{itemize}
		\item Versuch Anzahl Vorkommen und Länge gleichermaßen miteinzubeziehen
	\end{itemize}}
	\note[item]{Beginnen nur mit einer Startregel \begin{itemize}
		\item Die Regeln nummerieren wir durch.
		\item Beginnen bei $0$ (= Startregel)
	\end{itemize}}

	\begin{itemize}
		\item String $abacaba$
		\item Wähle als Flächenfunktion $A(i,j) = W(i,j) \cdot H(i,j)$.
		\item Beginne mit Grammatik: $R_0 \rightarrow abacaba\$$
	\end{itemize}
\end{frame}

\begin{frame}
	\frametitle{Beispiel}
	
	\note[item]{Beginnen mit dem Suffix- und LCP-Array}
	\note[item]{Beachten, dass das $\$$ an den String angehängt ist}
	\note[item]{Aus Einfachheitsgründen lassen wird den Kindintervallbaum hier weg}

	\note<2->[item]{Nehmen das LCP-Intervall mit größtem Flächenwert}
	\note<2->[item]{In diesem Fall $LCP[1..1]$ \begin{itemize}
		\item Höhe $3$, weil Minimum auf dem Intervall
		\item Breite $2$, weil das Intervall $2$ Vorkommen bedeutet
	\end{itemize}}
	\note<3->[item]{Entsprechendes $SA$-Intervall ist $SA[0..1]$ \begin{itemize}
		\item Weil LCP-Wert Länge des LCP der dieser zwei Indizes im $SA$ beschreibt
	\end{itemize}} 

	\note<4->[item]{Also Vorkommen der Länge $3$ bei Indizes $0$ und $4$ \begin{itemize}
		\item diese Beiden
	\end{itemize}}

	\begin{figure}[H]
		\centering
		\scalebox{0.7} {
			\begin{tikzpicture}[ampersand replacement=\&]
				\matrix (m) [matrix of nodes, 
					nodes={
						draw, 
						rectangle, 
						minimum width=7mm, 
						minimum height=7mm, 
						outer sep=0pt,
						inner sep=0,
						anchor=center
					}
				]{ 
					0 \& 1 \& 2 \& 3 \& 4 \& 5 \& 6 \& 7 \\
					a \& b \& a \& c \& a \& b \& a \& \$\\
					0 \& 4 \& 2 \& 6 \& 1 \& 5 \& 3 \& 7 \\ 
					0 \& 3 \& 1 \& 1 \& 0 \& 2 \& 0 \& 0 \\
				};
				
				\node[draw=none, left=of m-1-1] (i) {\textbf{i}}; 
				\node[draw=none, left=of m-2-1] (s) {\textbf{s\$}};
				\node[draw=none, left=of m-3-1] (SA) {\textbf{SA}};
				\node[draw=none, left=of m-4-1] (LCP) {\textbf{LCP}};

				%\pause

				\node<2, 4->[draw=red, inner sep=1.5mm, label=below:{$LCP[1..1]$}, fit=(m-4-2) (m-4-2)] {};
				\node<3>[draw=blue, inner sep=1.5mm, fit=(m-3-1) (m-3-2)] {};
				\node<3>[draw=none, inner sep=1.5mm, label=below:{$SA[0..1]$}, fit=(m-4-1) (m-4-2)] {};

				\node<4->[draw, inner sep=1.5mm, fit=(m-2-1) (m-2-3)] {};
				\node<4->[draw, inner sep=1.5mm, fit=(m-2-5) (m-2-7)] {};



			\end{tikzpicture}
		}
	\end{figure}
	
	\begin{itemize}
		\item<2-> $A(1,1) = W(1, 1) \cdot H(1, 1) = 2 \cdot 3 = 6$
		\item<5-> Erzeuge Produktionsregel $R_1 \rightarrow aba$ und ersetze die Vorkommen
	\end{itemize}
\end{frame}

\begin{frame}
	\frametitle{Beispiel}

	Neue Grammatik:
	\begin{align*}
		R_0 &\rightarrow R_1 c R_1\$\\
		R_1 &\rightarrow aba
	\end{align*}
	
\end{frame}

\subsection{Datenstrukturen}

\begin{frame}{Umsetzung}

	\note<1->[item]{Wenn Substrings durch Nichtterminale ersetzt werden: \begin{itemize}
		\item Manche späteren Ersetzungen werden unmöglich oder unnütz
		\item Da sie sich mit vorherigen Ersetzungen überschneiden
		\item Oder weil durch frühere Ersetzungen 
	\end{itemize}}
	\note<1->[item]{Beispiel von Gerade}
	\note<2->[item]{Ersetzung von $ac$ nicht möglich \begin{itemize}
		\item Auch wenn es in diesem Beispiel gerade keinen Sinn macht. nur hypothetisch
		\item $aba$ ja schon substituiert. Das $ac$ darf also nicht substituiert werden
		\item Überschneidet sich mit dem $a$ bei Index $2$
	\end{itemize}}
	\note<3->[item]{Nächster Fall: Ersetzung von $ba$ theoretisch möglich (Regel $R_2 \rightarrow ba$) aber unnütz \begin{itemize}
		\item Denn $ba$ kommt nur noch einmal in der Grammatik vor
		\item Grund: Substitution von $aba$
	\end{itemize}}
	\note<3->[item]{Brauchen also Datenstruktur, die Substitutionen speichert und diese Fälle erkennen kann}

	\begin{columns}
		\begin{column}{0.6\linewidth}
			\scalebox{0.7} {
				\begin{tikzpicture}[ampersand replacement=\&]
					\matrix (m) [matrix of nodes, 
						nodes={
							draw, 
							rectangle, 
							minimum width=7mm, 
							minimum height=7mm, 
							outer sep=0pt,
							inner sep=0,
							anchor=center
						}
					]{ 
						0 \& 1 \& 2 \& 3 \& 4 \& 5 \& 6 \& 7 \\
						a \& \only<1, 2>{b}\only<3->{\textcolor{red}{b}} \& \only<1>{a}\only<2->{\textcolor{red}{a}} \& c \& a \& \only<1, 2>{b}\only<3->{\textcolor{red}{b}} \& \only<1, 2>{a}\only<3->{\textcolor{red}{a}} \& \$\\
						0 \& 4 \& 2 \& 6 \& 1 \& 5 \& 3 \& 7 \\ 
						0 \& 3 \& 1 \& 1 \& 0 \& 2 \& 0 \& 0 \\
					};
					
					\node[draw=none, left=of m-1-1] (i) {\textbf{i}}; 
					\node[draw=none, left=of m-2-1] (s) {\textbf{s\$}};
					\node[draw=none, left=of m-3-1] (SA) {\textbf{SA}};
					\node[draw=none, left=of m-4-1] (LCP) {\textbf{LCP}};

					\node<2>[draw, inner sep=1.5mm, fit=(m-2-3) (m-2-4)] {};
					\node<3->[draw, inner sep=1.5mm, fit=(m-2-2) (m-2-3)] {};
					\node<3->[draw, inner sep=1.5mm, fit=(m-2-6) (m-2-7)] {};
					\node<3->[draw=red, inner sep=1.5mm, label=below:{$LCP[5..5]$}, fit=(m-4-6) (m-4-6)] {};
				\end{tikzpicture}
			}
		\end{column}
		\begin{column}{0.3\linewidth}
			\begin{align*}
				R_0 &\rightarrow R_1 c R_1\\
				R_1 &\rightarrow a \only<1,2>{b} \only<3->{\textcolor{red}{b}}   \only<1>{a} \only<2->{\textcolor{red}{a}}
			\end{align*}
		\end{column}
	\end{columns}
	\begin{columns}
		\begin{column}{\linewidth}
			\centering
			\only<2>{$ac$ überschneidet sich mit schon ersetztem Intervall}
			\only<3>{Durch vorherige Substitution: $ba$ nur noch einmal in der Grammatik}
		\end{column}
	\end{columns}
\end{frame}


\begin{frame}{Ersetzungsintervall}
	
	\note<1->[item]{Erinnerung: Regeln werden durchnummeriert} 
	\note<2->[item]{Zum Speichern von Substitutionen: Ersetzungsintervalle \begin{itemize}
		\item Es wird also gespeichert, durch welche Regel ein bestimmtes Intervall ersetzt wurde 
	\end{itemize}}

	\begin{block}<2->{Ersetzungsintervall}
		\begin{itemize}
			\item Eingabestring $s = s_0, \dots, s_{n-1}$ und $R_k$ das Nichtterminal der Regel $k$.
			\item Wird ein Intervall $[i..j]$ durch das Nichtterminal $R_k$ ersetzt, dann heißt $R_k$-$[i..j]$ Ersetzungsintervall der Regel $k$ bei Index $i$
		\end{itemize}
	\end{block}
\end{frame}

\begin{frame}{\texttt{RuleIntervalIndex}}

	\note[item]{Beispiel \begin{itemize}
		\item Folgende Grammatik
	\end{itemize}}
	\note[item]{Solch eine Datenstruktur wollen wir erreichen. \begin{itemize}
		\item Ersetzungsintervalle entsprechend ihrer Verschachtelung gespeichert
		\item sollte möglichst effizient modifizierbar und navigierbar sein
	\end{itemize}}

	\begin{columns}
		\begin{column}{0.15\linewidth}
			\begin{align*}
				R_0 &\rightarrow R_1 c R_1\$\\
				R_1 &\rightarrow a R_2\\
				R_2 &\rightarrow ba
			\end{align*}
		\end{column}
		\begin{column}{0.8\linewidth}
			\centering
			\begin{tikzpicture}[ampersand replacement=\&]
				\matrix (m) [matrix of nodes, 
					nodes={
						draw=none, 
						rectangle, 
						minimum width=7mm, 
						minimum height=7mm, 
						outer sep=0pt,
						inner sep=0,
						anchor=center,
						align=center
					},
					nodes in empty cells
				]{ 
					0 \& 1 \& 2 \& 3 \& 4 \& 5 \& 6 \& 7 \\
					a \& b \& a \& c \& a \& b \& a \& \$\\
					\& \& \& \& \& \& \&  \\
					\& \& \& \& \& \& \&  \\
					\& \& \& \& \& \& \&  \\
				};
				\node[draw, inner sep=0, fit=(m-3-1) (m-3-8), text height=1.25em] {$R_0$-$[0..7]$};

				\node[draw, inner sep=0, fit=(m-4-1) (m-4-3), text height=1.25em] {$R_1$-$[0..2]$};
				\node[draw, inner sep=0, fit=(m-4-5) (m-4-7), text height=1.25em] {$R_1$-$[4..6]$};
		
				\node[draw, inner sep=0, fit=(m-5-2) (m-5-3), text height=1.25em] {$R_2$-$[1..2]$};
				\node[draw, inner sep=0, fit=(m-5-6) (m-5-7), text height=1.25em] {$R_2$-$[5..6]$};
			\end{tikzpicture}
		\end{column}
	\end{columns}
\end{frame}

\begin{frame}{\texttt{RuleIntervalIndex}}

	\note<1->[item]{Basiert auf assoziativer Predecessor-Datenstruktur \begin{itemize}
		\item Assoziativ: Speichern nicht nur Schlüssel. Jeder Schlüssel hat zugehörige Daten
	\end{itemize}}
	\note<2->[item]{Speichert für einen Index immer das am tiefsten verschachtelte Intervall, das an diesem Index beginnt \begin{itemize}
		\item falls kein Intervall an diesem Index beginnt, ist der Eintrag leer
	\end{itemize}}

	\begin{itemize}
		\item<1-> Basiert auf assoziativer Predecessor-Datenstruktur \begin{itemize}
			\item Daten: Ersetzungsintervalle
			\item Schlüssel: Startindizes der Intervalle
		\end{itemize}
		\item<2-> Speichert für einen Index das tiefste Intervall an diesem Startindex, falls vorhanden
		\item<3-> Restliche Intervalle mit Pointern verbunden
	\end{itemize}

\end{frame}

\begin{frame}{\texttt{RuleIntervalIndex}}

	\note<1->[item]{intervalContaining: Erhält zwei Intervallgrenzen als Eingabe. Gibt das tiefste 	verschachtelte Ersetzungsintervall in der Datenstruktur zurück, das $[i..j]$ enthält.}
	\note<2->[item]{mark: Fügt ein neues Ersetzungsintervall ein \begin{itemize}
		\item Markiert sogesehen ein Intervall als \emph{ersetzt}
	\end{itemize}}
	\note<3->[item]{get: Das tiefste Intervall an diesem Startindex}

	Funktionen: \begin{itemize}
		\item<1-> \texttt{intervalContaining(i,j)}\\Das tiefste Intervall, das $[i..j]$ umschließt
		\item<2-> \texttt{mark(id, start, end)}\\Fügt ein neues Ersetzungsintervall $R_{id}$-$[start..end]$ ein
		\item<3-> \texttt{get(i)}\\Das tiefste Intervall zurück, das bei $i$ beginnt, falls vorhanden
	\end{itemize}
\end{frame}

\begin{frame}{Bereinigung der Vorkommen}

	\note<1-3>[item]{Problem von vorher: Vielleicht nicht alle Vorkommen ersetzbar}
	\note<2-3>[item]{Wie feststellen, ob ein Vorkommen ersetzt werden darf?}
	\note<3>[item]{Beispiel von vorhin}
	\note<4>[item]{für $abac$ \begin{itemize}
		\item Folgende Visualisierung: \begin{itemize}
			\item Ziehe Pfeile von Unten bei den Grenzen des zu ersetzenden Substrings
			\item Pfeile dürfen nicht durch Ersetzungsintervalle hindurch
			\item Dürfen aber an Intervallgrenzen vorbei, die der Pfeil \emph{berührt}
			\item Kann ein Intervall gefunden werden, das beide Pfeile berühren, ist die Ersetzung erlaubt
			\item sonst nicht
		\end{itemize}
		\item Solch ein Intervall ist gefunden ($0..7$)
		\item Also substitution erlaubt
		\item Schaut man auf Grammatik, Substitution ist möglich
	\end{itemize}}
	\note<5>[item]{Wir versuchen $ac$ ersetzen \begin{itemize}
		\item Schaue auf Grammatik: nicht möglich
		\item hier also $ac$ nicht ersetzbar
	\end{itemize}}
	

	\only<2>{Wie feststellen, ob ein Vorkommen ersetzt werden darf?}
	\only<4>{z.B versuchen $abac$ zu ersetzen}
	\only<5>{z.B versuchen $ac$ zu ersetzen}

	\centering
	\only<3-> {
		\begin{columns}
			\begin{column}{0.15\linewidth}
				\centering
				\begin{align*}
					R_0 &\rightarrow R_1 c R_1\$\\
					R_1 &\rightarrow a R_2\\
					R_2 &\rightarrow ba
				\end{align*}
			\end{column}
			\begin{column}{0.8\linewidth}
				\centering
				\begin{tikzpicture}[ampersand replacement=\&]
					\matrix (m) [matrix of nodes, 
						nodes={
							draw=none, 
							rectangle, 
							minimum width=7mm, 
							minimum height=7mm, 
							outer sep=0pt,
							inner sep=0,
							anchor=center,
							align=center
						},
						nodes in empty cells
					]{ 
						0 \& 1 \& 2 \& 3 \& 4 \& 5 \& 6 \& 7 \\
						a \& b \& a \& c \& a \& b \& a \& \$\\
						\& \& \& \& \& \& \&  \\
						\& \& \& \& \& \& \&  \\
						\& \& \& \& \& \& \&  \\
					};
					\node<3>[draw, inner sep=0, fit=(m-3-1) (m-3-8), text height=1.25em] (0-7) {$R_0$-$[0..7]$};
			
					\node[draw, inner sep=0, fit=(m-4-1) (m-4-3), text height=1.25em] (0-2) {$R_1$-$[0..2]$};
					\node[draw, inner sep=0, fit=(m-4-5) (m-4-7), text height=1.25em] (4-6) {$R_1$-$[4..6]$};
				
					\node<3>[draw, inner sep=0, fit=(m-5-2) (m-5-3), text height=1.25em] (1-2) {$R_2$-$[1..2]$};
					\node[draw, inner sep=0, fit=(m-5-6) (m-5-7), text height=1.25em] (5-6){$R_2$-$[5..6]$};
					
					\only<5-> {
						\draw[->] ([yshift=-5mm]1-2.south) -- (1-2.south);
						\draw[->] ([yshift=-5mm]m-1-4.east|-1-2.south) -- (m-1-4.east|-0-7.south);
						\node[draw=red, fit=(m-2-3) (m-2-4), inner sep=0] {};
						\node[draw=blue, inner sep=0, fit=(m-3-1) (m-3-8), text height=1.25em] (0-7) {$R_0$-$[0..7]$};
						\node[draw=blue, inner sep=0, fit=(m-5-2) (m-5-3), text height=1.25em] (1-2) {$R_2$-$[1..2]$};
					}
						
					\only<4> {
						\draw[->] ([yshift=-5mm]0-7.west|-1-2.south) -- (0-7.west|-0-7.south);
						\draw[->] ([yshift=-5mm]m-1-4.east|-1-2.south) -- (m-1-4.east|-0-7.south);
						\node[draw=red, fit=(m-2-1) (m-2-4), inner sep=0] {};
						\node[draw=blue, inner sep=0, fit=(m-3-1) (m-3-8), text height=1.25em] (0-7) {$R_0$-$[0..7]$};
						\node[draw, inner sep=0, fit=(m-5-2) (m-5-3), text height=1.25em] (1-2) {$R_2$-$[1..2]$};
					}
					
				\end{tikzpicture}
				
				\only<4>{Ersetzung möglich: $R_3 \rightarrow R_1 c$}
				\only<5>{Ersetzung nicht möglich}
			\end{column}
		\end{columns}
	}
\end{frame}