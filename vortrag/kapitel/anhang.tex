\section{Anhang}

\begin{frame}{\texttt{RuleIntervalIndex}}

	Zusätzliche Daten für Ersetzungsintervalle: \begin{itemize}[leftmargin=3cm]
		\item<2->\texttt{parent}\\Das tiefste Intervall, das dieses Intervall umschließt
		\item<3->\texttt{firstAtStartIndex}\\Das am wenigsten verschachtelte Intervall am selben Startindex
		\item<4->\texttt{nextAtStartIndex}\\Das nächsttiefer verschachtelte Intervall am selben Startindex
	\end{itemize}

\end{frame}



\begin{frame}{\texttt{RuleIntervalIndex}}

	\note<-4>[item]{Beispiel \begin{itemize}
		\item Folgende Grammatik
	\end{itemize}}
	\note<-4>[item]{Solch eine Datenstruktur wollen wir erreichen. \begin{itemize}
		\item Ersetzungsintervalle entsprechend ihrer Verschachtelung gespeichert
		\item sollte möglichst effizient modifizierbar und navigierbar sein
	\end{itemize}}
	\note<2-4>[item] {parent Pointer \begin{itemize}
		\item jetzt Rot
		\item das nächst übergeordnete Intervall
	\end{itemize}}
	\note<3-4>[item] {firstAtStartIndex \begin{itemize}
		\item Das am wenigsten verschachtelte Intervall, das am selben Startindex beginnt
		\item Pointer von $[0..2]$ Zeigt auf $[0..7]$, da es am selben Startindex beginnt und am wenigsten verschachtelt ist
		\item Alle anderen Zeigen auf sich selbst
	\end{itemize}}
	\note<4>[item]{nextAtStartIndex \begin{itemize}
		\item Nur bei $[0..7]$ belegt
		\item Zeigt auf $[0..2]$, da es am selben Startindex beginnt und nächsttiefer verschachtelt ist 
	\end{itemize}}
	\note<5->[item]{Intervalle die in der Predecessor Struktur gespeichert werden: \begin{itemize}
		\item rot markiert
		\item immer die tiefsten Intervalle an diesem Startindex
		\item Am wenigsten verschachteltes Intervall an diesem Index durch \texttt{firstAtStartIndex}
		\item Rest durch \texttt{parent} und \texttt{nextAtStartIndex} erreichbar
	\end{itemize}}

	\begin{columns}
		\begin{column}{0.15\linewidth}
			\centering
			\begin{align*}
				R_0 &\rightarrow R_1 c R_1\$\\
				R_1 &\rightarrow a R_2\\
				R_2 &\rightarrow ba
			\end{align*}
			\only<2>{\textcolor{red}{\texttt{parent}}}
			\only<3>{\textcolor{red}{\texttt{firstAtStartIndex}}}
			\only<4>{\textcolor{red}{\texttt{nextAtStartIndex}}}
		\end{column}
		\begin{column}{0.8\linewidth}
			\centering
			\begin{tikzpicture}[ampersand replacement=\&]
				\matrix (m) [matrix of nodes, 
					nodes={
						draw=none, 
						rectangle, 
						minimum width=7mm, 
						minimum height=7mm, 
						outer sep=0pt,
						inner sep=0,
						anchor=center,
						align=center
					},
					nodes in empty cells
				]{ 
					0 \& 1 \& 2 \& 3 \& 4 \& 5 \& 6 \& 7 \\
					a \& b \& a \& c \& a \& b \& a \& \$\\
					\& \& \& \& \& \& \&  \\[8mm]
					\& \& \& \& \& \& \&  \\[8mm]
					\& \& \& \& \& \& \&  \\
				};
				\node[draw, inner sep=0, fit=(m-3-1) (m-3-8), text height=1.25em] (0-7) {$R_0$-$[0..7]$};
		
				\only<1-4> {
					\node[draw, inner sep=0, fit=(m-4-1) (m-4-3), text height=1.25em] (0-2) {$R_1$-$[0..2]$};
					\node[draw, inner sep=0, fit=(m-4-5) (m-4-7), text height=1.25em] (4-6) {$R_1$-$[4..6]$};
				
					\node[draw, inner sep=0, fit=(m-5-2) (m-5-3), text height=1.25em] (1-2) {$R_2$-$[1..2]$};
					\node[draw, inner sep=0, fit=(m-5-6) (m-5-7), text height=1.25em] (5-6){$R_2$-$[5..6]$};
				}
				\only<5-> {
					\node[draw=red, inner sep=0, fit=(m-4-1) (m-4-3), text height=1.25em] (0-2) {$R_1$-$[0..2]$};
					\node[draw=red, inner sep=0, fit=(m-4-5) (m-4-7), text height=1.25em] (4-6) {$R_1$-$[4..6]$};
				
					\node[draw=red, inner sep=0, fit=(m-5-2) (m-5-3), text height=1.25em] (1-2) {$R_2$-$[1..2]$};
					\node[draw=red, inner sep=0, fit=(m-5-6) (m-5-7), text height=1.25em] (5-6){$R_2$-$[5..6]$};
				}
				
				\only<1, 3-> {
					\draw[->] (0-2.north) |- (0-2.north|-0-7.south);
					\draw[->] (4-6.north) |- (4-6.north|-0-7.south);
					\draw[->] (1-2.north) |- (1-2.north|-0-2.south);
					\draw[->] (5-6.north) |- (5-6.north|-4-6.south);
				}
				\only<2> {
					\draw[->, draw=red] (0-2.north) |- (0-2.north|-0-7.south);
					\draw[->, draw=red] (4-6.north) |- (4-6.north|-0-7.south);
					\draw[->, draw=red] (1-2.north) |- (1-2.north|-0-2.south);
					\draw[->, draw=red] (5-6.north) |- (5-6.north|-4-6.south);
				}

				\only<1-2, 4-> {
					\draw[->] (0-2.west) to[out=180, in=180] ([yshift=-2mm]0-7.west);
					\draw[->] ([yshift=-1mm]0-7.west) to[out=180, in=180, looseness=3] ([yshift=3mm]0-7.west);
					\draw[->] ([yshift=-2mm]1-2.west) to[out=180, in=180, looseness=3] ([yshift=2mm]1-2.west);
					\draw[->] ([yshift=-2mm]4-6.west) to[out=180, in=180, looseness=3] ([yshift=2mm]4-6.west);
					\draw[->] ([yshift=-2mm]5-6.west) to[out=180, in=180, looseness=3] ([yshift=2mm]5-6.west);
				}
				\only<3> {
					\draw[->, draw=red] (0-2.west) to[out=180, in=180] ([yshift=-2mm]0-7.west);
					\draw[->, draw=red] ([yshift=-1mm]0-7.west) to[out=180, in=180, looseness=3] ([yshift=3mm]0-7.west);
					\draw[->, draw=red] ([yshift=-2mm]1-2.west) to[out=180, in=180, looseness=3] ([yshift=2mm]1-2.west);
					\draw[->, draw=red] ([yshift=-2mm]4-6.west) to[out=180, in=180, looseness=3] ([yshift=2mm]4-6.west);
					\draw[->, draw=red] ([yshift=-2mm]5-6.west) to[out=180, in=180, looseness=3] ([yshift=2mm]5-6.west);
				}

				\draw<1-3>[<-] (0-2.west|-0-2.north) |- (0-2.west|-0-7.south);
				\draw<4>[<-, draw=red] (0-2.west|-0-2.north) |- (0-2.west|-0-7.south);
			\end{tikzpicture}
		\end{column}
	\end{columns}
\end{frame}