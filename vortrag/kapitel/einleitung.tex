\section{Einleitung}

\begin{frame}
	\frametitle{Textkompression}

	\note<1>[item]{Textkompression: Kodierung von Text in Repräsentation mit weniger Bits als eingabe \begin{itemize}
		\item Verlustfrei: Eingabe fehlerlos wiederherstellbar
	\end{itemize}}
	\note<2>[item]{Steigende Menge von Daten}
	\note<2>[item]{Für Übertragung etc. Kompression nötig}
	\note<3>[item] {
		Entropie ordnet jedem Zeichen eine unterschiedliche Folge von Bits zu. beispiel Huffman-Kodierung
	}
	\note<4>[item] {
		Wörterbuchkompression 
		\begin{itemize}
			\item im Eingabetext nach wiederholten Substrings suchen
			\item Diese durch einen Pointer zu einem Wörterbucheintrag ersetzen
		\end{itemize}
	}
	\note<5>[item] {
		Grammatikkompression 
		\begin{itemize}
			\item erzeugt Kontextfreie Grammatik, die Eingabestring erzeugt
			\item möglichst klein
			\item gleich genauer definiert
			\item Grammatik muss zum Abspeichern noch kodiert werden \begin{itemize}
				\item nicht Teil dieser Arbeit
			\end{itemize}
			\item bsp. Sequitur \begin{itemize}
				\item On-Line Algorithmus (liest Eingabe nach und nach) 
				\item Baut inkrementell eine Grammatik, die bestimmte Eigenschaften erfüllt
			\end{itemize}
			\item bsp. RePair \begin{itemize}
				\item Off-Line Algorithmus (von Anfang an Eingabe im Speicher)
				\item Sucht immer am meisten wiederholtes Symbolpaar und ersetzt es durch Nichtterminal
			\end{itemize}
			\item beide Algorithmen $\mathcal{O}(n)$
		\end{itemize}
	}

	\note<6>[item]{
		Diese Arbeit: AreaComp \begin{itemize}
			\item Grammatikkompressionsalgorithmus
			\item arbeitet Off-Line $\Rightarrow$ Gesamte Eingabe zu Anfang im Speicher
			\item nutzt Enhanced Suffix Array \begin{itemize}
				\item später Erklärung
			\end{itemize}
		\end{itemize} 
	}

	\begin{itemize}
		\item<1-> Textkompression: Text in weniger Bits darstellen als Eingabe
		\item<2-> Verschiedene Ansätze für Textkompression 
		\begin{itemize}
			\item<3-> Entropie-Kodierer (z.B. Huffman \cite{huffman_method_1952})
			\item<4-> Wörterbuchkompression (z.B. LZ77 \cite{ziv_universal_1977})
			\item<5-> \textbf{Grammatikkompression} (z.B. Sequitur \cite{nevill-manning_identifying_1997} und RePair \cite{larsson_off-line_2000})
		\end{itemize}
		\item<6-> Grammatikkompressionsalgorithmus AreaComp \begin{itemize}
			\item Off-Line
			\item Nutzt Enhanced Suffix-Array
		\end{itemize}
	\end{itemize}
\end{frame}