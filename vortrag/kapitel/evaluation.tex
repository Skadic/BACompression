%\addtocontents{toc}{\newpage}

\pgfplotscreateplotcyclelist{algs}{%
    lime!80!black,every mark/.append style={fill=lime},mark=square*\\%
    red,every mark/.append style={fill=red},mark=triangle*\\%
    teal,every mark/.append style={fill=teal!80},mark=pentagon*\\%
    black,mark=star\\%
    purple!30!blue,every mark/.append style={fill=purple!30!blue},mark=diamond*\\%
    orange!80!black,every mark/.append style={solid,fill=orange},mark=*\\%
}

\pgfplotscreateplotcyclelist{algso}{%
    lime!80!black,fill=lime\\%
    red,fill=red\\%
    teal,fill=teal!80\\%
    black,fill=black!80\\%
    purple!30!blue,fill=purple!30!blue\\%
    orange!80!black,fill=orange\\%
    cyan!80!black,fill=cyan\\%
}

\pgfplotsset{every tick label/.append style={font=\tiny}}

\pgfplotsset{
    dataplot/.style={
        xtick={1,2,3,4,5},
        xticklabels={10,20,30,40,50},
        xticklabel style={align=center},
        %xlabel={Größe der Eingabe},
        %x unit={MB},
        width=37mm,
        height=45mm,
		y label style={at={(axis description cs:-0.25,.5)}, anchor=south},
		x label style={at={(axis description cs:1.2,-0.1)},anchor=north},
        legend style={at={(2.5,-0.4)}, anchor=north},
        cycle list name=algs,
		legend columns=3, 
		legend style={
			font=\tiny,
			% the /tikz/ prefix is necessary here...
			% otherwise, it might end-up with `/pgfplots/column 2`
			% which is not what we want. compare pgfmanual.pdf
			/tikz/column 2/.style={
				column sep=5pt,
			},
			/tikz/column 3/.style={
				column sep=5pt,
			},
		}
    },
    runtimeplot/.style={
        dataplot,
        %y unit={s},
        %ylabel={Laufzeit},
        ytick distance=100
    },
    sizeplot/.style={
        dataplot,
        %y unit={outputsize/inputsize},
        %ylabel={Größenverhältnis},
        ytick distance=0.1
    },
    depthplot/.style={
        dataplot,
        %y unit={\#Schritte zur tiefsten Regel},
        %ylabel={Tiefe}
    },
    countplot/.style={
        dataplot,
        %ylabel={Regeln}
    },
    lengthplot/.style={
        dataplot,
        %y unit={\#Zeichen},
        %ylabel={Durchschnittslänge}
    },
	noylabel/.style={
		ylabel={},
        y unit={},
		ytick style={draw=none}
	},
	sizexlabel/.style={
		xlabel={Größe der Eingabe},
		x unit={MB}
	},
	rightdist/.style={
		anchor=west, xshift=6mm,
	}
}

\section{Evaluation}

\subsection{Methodik}

\begin{frame}
	\frametitle{Algorithmen}

	\note[item]{3 Algorithmen}
	\note<2->[item]{Sequitur \begin{itemize}
		\item Java-Implementierung von Professor Eibe Frank
	\end{itemize}}
	\note<3->[item]{RePair \begin{itemize}
		\item Eigene Implementierung
		\item Keine für diese Zwecke angemessene JVM-Implementierung gefunden
	\end{itemize}}
	\note<4->[item]{AreaComp \begin{itemize}
		\item 4 Flächenfunktionen
		\item betrachten wir im Folgenden
	\end{itemize}}
	\note<4->[item]{Alle Algorithmen in Java 16}

	\begin{itemize}
		\item<2-> Sequitur
		\item<3-> RePair
		\item<4-> AreaComp \begin{itemize}
			\item 4 Flächenfunktionen 
		\end{itemize}
	\end{itemize}
\end{frame}

\begin{frame}
	\frametitle{Flächenfunktionen}

	\note<1-4>[item]{Wie gesagt, 4 Flächenfunktionen für AreaComp}

	\note<2-4>[item]{ChildArea \begin{itemize}
		\item Wie im AreaComp Beispiel
		\item Versucht durch die Multiplikation die erreichbare Kompression abzuschätzen
		\item Heißt so, weil während der Entwicklung dies die erste Flächenfunktion war, die Kindintervalle benutzt, um Höhe zu berechnen
	\end{itemize}}
	\note<3-4>[item]{WidthFirstArea \begin{itemize}
		\item Priorisiert die breitesten LCP-Intervalle
	\end{itemize}}
	\note<4-4>[item]{HeightFirstArea \begin{itemize}
		\item Priorisiert die höchsten LCP-Intervalle
		\item Also die Intervalle mit längsten Substring
	\end{itemize}}
	\note<5->[item]{HeightAdvantageArea \begin{itemize}
		\item Versuch einer Verbesserung von HeightFirstArea
		\item Da HeightFirstArea in Tests gut funktioniert hat
		\item schaut auch zum Teil nach Breite des Intervalls 
	\end{itemize}}
	
	\begin{itemize}
		\item<2-> \texttt{ChildArea}: $A(i,j) = H(i,j) \cdot W(i,j)$
		\item<3-> \texttt{WidthFirstArea}: $A(i,j) = W(i,j)$
		\item<4-> \texttt{HeightFirstArea}: $A(i,j) = H(i,j)$
		\item<5-> \texttt{HeightAdvantageArea}: $A(i,j) = 10 \cdot H(i,j) + \min\{\ln(W(i,j)), 9\}$
	\end{itemize}
\end{frame}


\begin{frame}
	\frametitle{Datensätze}
	\centering

	\note<1->[item]{Haben Algorithmen, brauchen also noch Testdaten}
	\note<1->[item]{4 Datensätze aus dem Pizza\&Chili Corpus \begin{itemize}
		\item Unterschiedliche Alphabetgröße
		\item Präfixe der Datensätze $10$ bis $50$ Megabyte. In $10$ MB Schritten
	\end{itemize}}
	\note<1->[item]{Verschiedene Alphabetgrößen}
	\note<2->[item]{DNA Datensatz \begin{itemize}
		\item enthält DNA Sequenzen
	\end{itemize}}
	\note<3->[item]{English Datensatz \begin{itemize}
		\item enthält englischen Text
	\end{itemize}}
	\note<4->[item]{Proteins Datensatz \begin{itemize}
		\item enthält Proteinsequenzen
	\end{itemize}}
	\note<5->[item]{XML Datensatz \begin{itemize}
		\item xml Datei mit Informationen über Informatik-Zeitschriften
	\end{itemize}}


	\begin{tabular}{l|c}
		Datensatz & Alphabetgröße\\ \hline
		\pause
		\emph{dna} & $16$\\
		\pause
		\emph{english} & $239$\\
		\pause
		\emph{proteins} & $27$\\
		\pause
		\emph{xml} & $97$ 
	\end{tabular}

\end{frame}


\subsection{Ergebnisse}

\begin{frame}
	\frametitle{Laufzeit}
	\note<1>[item]{Ursprüngliche grobe Abschätzung: $\mathcal{O}(n^2 \cdot d)$}
	\note[item]{Laufzeit in Sekunden}
	\note[item]{Sequitur auf allen Datensätzen am schnellsten}
	\note[item]{Nächstschnellster Algorithmus: RePair}
	\note[item]{Wahl der Flächenfunktion \begin{itemize}
		\item Keine allzu große Auswirkung
		\item Auf XML und Proteins quasi identisch
		\item Auf DNA, WidthFirst und Child etwas schneller, auf English nur WidthFirst etwas schneller
	\end{itemize}}
	\note[item]{Trotz schlechterer Abschätzung scheint Laufzeit linear zu steigen}

	\only<1>{Ursprüngliche Laufzeitabschätzung: $\mathcal{O}(n^2 \cdot d)$ \begin{itemize}
		\item $d =$ Tiefe der Grammatik wenn der Algorithmus terminiert
	\end{itemize}}

	\only<2>{
		\begin{columns}
			\column{\dimexpr\paperwidth-10pt}
	
			\begin{tikzpicture}
				\begin{axis}[runtimeplot, title={DNA}, name=dnatime, y unit={s}, ylabel={Laufzeit}]
				   
					%% MULTIPLOT(algo) SELECT SUBSTR(datasetsize, 1, 1) AS x, AVG((comptime + unifytime) / 1000) AS y, MULTIPLOT
					%% FROM dna GROUP BY MULTIPLOT, datasetsize
					\addplot coordinates { (1,47.0) (2,106.0) (3,171.0) (4,232.0) (5,314.0) };
					\addplot coordinates { (1,51.0) (2,116.0) (3,198.0) (4,282.0) (5,380.0) };
					\addplot coordinates { (1,50.0) (2,117.0) (3,190.0) (4,266.0) (5,368.0) };
					\addplot coordinates { (1,44.0) (2,103.0) (3,159.0) (4,227.0) (5,306.0) };
					\addplot coordinates { (1,15.0) (2,35.0) (3,57.0) (4,75.0) (5,102.0) };
					\addplot coordinates { (1,4.0) (2,9.0) (3,15.0) (4,23.0) (5,31.0) };
					
					\addlegendentry{ChildArea};
					\addlegendentry{HeightAdvantageArea};
					\addlegendentry{HeightFirstArea};
					\addlegendentry{WidthFirstArea};
					\addlegendentry{RePair};
					\addlegendentry{Sequitur};
				\end{axis}
		
				\begin{axis}[runtimeplot, title={Proteins}, at=(dnatime.east), rightdist, name=proteinstime, sizexlabel]
					%% MULTIPLOT(algo) SELECT SUBSTR(datasetsize, 1, 1) AS x, AVG((comptime + unifytime) / 1000) AS y, MULTIPLOT
					%% FROM proteins GROUP BY MULTIPLOT, datasetsize
					\addplot coordinates { (1,43.0) (2,97.0) (3,135.0) (4,167.0) (5,214.0) };
					\addplot coordinates { (1,40.0) (2,87.0) (3,125.0) (4,159.0) (5,202.0) };
					\addplot coordinates { (1,39.0) (2,86.0) (3,131.0) (4,161.0) (5,205.0) };
					\addplot coordinates { (1,39.0) (2,84.0) (3,118.0) (4,154.0) (5,192.0) };
					\addplot coordinates { (1,16.0) (2,42.0) (3,79.0) (4,92.0) (5,123.0) };
					\addplot coordinates { (1,6.0) (2,15.5) (3,27.0) (4,40.0) (5,58.0) };
				\end{axis}
	
				\begin{axis}[runtimeplot, title={XML}, at=(proteinstime.east), rightdist, name=xmltime]
					%% MULTIPLOT(algo) SELECT SUBSTR(datasetsize, 1, 1) AS x, AVG((comptime + unifytime) / 1000) AS y, MULTIPLOT
					%% FROM xml GROUP BY MULTIPLOT, datasetsize
					\addplot coordinates { (1,41.0) (2,91.0) (3,145.0) (4,203.0) (5,269.0) };
					\addplot coordinates { (1,35.0) (2,83.0) (3,135.0) (4,202.0) (5,265.0) };
					\addplot coordinates { (1,35.0) (2,85.0) (3,139.0) (4,192.0) (5,250.0) };
					\addplot coordinates { (1,35.0) (2,78.0) (3,125.0) (4,174.0) (5,232.0) };
					\addplot coordinates { (1,11.0) (2,29.0) (3,45.0) (4,63.0) (5,80.0) };
					\addplot coordinates { (1,4.0) (2,9.0) (3,15.0) (4,22.0) (5,29.0) };
				\end{axis}
		
				\begin{axis}[runtimeplot, title={English}, at=(xmltime.east), rightdist]
					%% MULTIPLOT(algo) SELECT SUBSTR(datasetsize, 1, 1) AS x, AVG((comptime + unifytime) / 1000) AS y, MULTIPLOT
					%% FROM english GROUP BY MULTIPLOT, datasetsize
					\addplot coordinates { (1,44.0) (2,113.0) (3,162.0) (4,231.0) (5,297.0) };
					\addplot coordinates { (1,47.0) (2,114.0) (3,174.0) (4,244.0) (5,315.0) };
					\addplot coordinates { (1,46.0) (2,114.0) (3,177.0) (4,242.0) (5,333.0) };
					\addplot coordinates { (1,37.0) (2,89.0) (3,138.0) (4,192.0) (5,246.0) };
					\addplot coordinates { (1,30.0) (2,77.0) (3,121.0) (4,168.0) (5,209.0) };
					\addplot coordinates { (1,5.0) (2,13.0) (3,23.0) (4,34.0) (5,47.0) };
				\end{axis}
			\end{tikzpicture}
		  \end{columns}
	}
\end{frame}

\begin{frame}
	\frametitle{Größenverhältnis der Größe der Grammatik zur Eingabe}

	\note<1>[item]{Betrachten hier Größenverhältnis der Größe der Grammatik zur Länge der Eingabe}
	\note<1>[item]{RePair produziert größte Grammatiken mit in der Regel großem Abstand (insbes. Proteins und English)}
	\note<1>[item]{Nächstkleinere: WidthFirstArea \begin{itemize}
		\item Unterschied zu RePair variiert
	\end{itemize}}
	\note<2>[item]{Wiederum nächstkleinere: ChildArea \begin{itemize}
		\item Unterschied zu WidthFirstArea variiert
	\end{itemize}}
	\note<2>[item]{HeightFirstArea und HeightAdvantageArea quasi identische Ergebnisse \begin{itemize}
		\item Auf XML und English fast so klein wie Grammatiken von Sequitur
	\end{itemize}}

	\note[item]{Sequitur produziert auf jedem Datensatz die kleinste Grammatik}


	\begin{columns}
		\column{\dimexpr\paperwidth-10pt}

		\begin{tikzpicture}
			\begin{axis}[sizeplot, title={DNA}, name=dnatime, ylabel={Größenverhältnis}]
			   
                \addplot coordinates { (1,0.429266) (2,0.429114) (3,0.429129) (4,0.429873) (5,0.429092) };
                \addplot coordinates { (1,0.216447) (2,0.205636) (3,0.201329) (4,0.197742) (5,0.194624) };
                \addplot coordinates { (1,0.218332) (2,0.207557) (3,0.203194) (4,0.199581) (5,0.19631) };
                \addplot coordinates { (1,0.429441) (2,0.429373) (3,0.429393) (4,0.430037) (5,0.429272) };
                \addplot coordinates { (1,0.647652) (2,0.527011) (3,0.640813) (4,0.53727) (5,0.516707) };
                \addplot coordinates { (1,0.146958) (2,0.139801) (3,0.138794) (4,0.137584) (5,0.135546) };
				
				\addlegendentry{ChildArea};
				\addlegendentry{HeightAdvantageArea};
				\addlegendentry{HeightFirstArea};
				\addlegendentry{WidthFirstArea};
				\addlegendentry{RePair};
				\addlegendentry{Sequitur};
			\end{axis}
	
			\begin{axis}[sizeplot, title={Proteins}, at=(dnatime.east), rightdist, name=proteinstime, sizexlabel]
				\addplot coordinates { (1,0.341726) (2,0.365969) (3,0.386289) (4,0.400567) (5,0.404739) };
                \addplot coordinates { (1,0.212851) (2,0.202009) (3,0.222857) (4,0.249495) (5,0.264087) };
                \addplot coordinates { (1,0.21661) (2,0.20566) (3,0.227152) (4,0.254638) (5,0.269725) };
                \addplot coordinates { (1,0.433566) (2,0.432746) (3,0.432112) (4,0.43228) (5,0.432293) };
                \addplot coordinates { (1,0.767845) (2,0.776813) (3,0.744482) (4,0.834881) (5,0.828773) };
                \addplot coordinates { (1,0.179865) (2,0.169194) (3,0.180369) (4,0.198115) (5,0.208265) };
			\end{axis}
			
			\begin{axis}[sizeplot, title={XML}, at=(proteinstime.east), rightdist, name=xmltime]
				\addplot coordinates { (1,0.150608) (2,0.153637) (3,0.157878) (4,0.157854) (5,0.166147) };
                \addplot coordinates { (1,0.086619) (2,0.0805638) (3,0.078124) (4,0.0753391) (5,0.0757765) };
                \addplot coordinates { (1,0.088598) (2,0.0824666) (3,0.0799382) (4,0.0771623) (5,0.0775188) };
                \addplot coordinates { (1,0.29432) (2,0.29514) (3,0.298642) (4,0.30164) (5,0.366257) };
                \addplot coordinates { (1,0.266995) (2,0.277118) (3,0.258805) (4,0.280782) (5,0.304951) };
                \addplot coordinates { (1,0.0779639) (2,0.0725272) (3,0.0702591) (4,0.0670736) (5,0.0680445) };
			\end{axis}
	
			\begin{axis}[sizeplot, title={English}, at=(xmltime.east), rightdist]
				\addplot coordinates { (1,0.281303) (2,0.266864) (3,0.29072) (4,0.305588) (5,0.314204) };
                \addplot coordinates { (1,0.14763) (2,0.124187) (3,0.134434) (4,0.140658) (5,0.139948) };
                \addplot coordinates { (1,0.152299) (2,0.128113) (3,0.138901) (4,0.145343) (5,0.14459) };
                \addplot coordinates { (1,0.372798) (2,0.374225) (3,0.382871) (4,0.388644) (5,0.389639) };
                \addplot coordinates { (1,0.671124) (2,0.582472) (3,0.5846) (4,0.653665) (5,0.60334) };
                \addplot coordinates { (1,0.134138) (2,0.116564) (3,0.12048) (4,0.122204) (5,0.120246) };
			\end{axis}
		\end{tikzpicture}
	  \end{columns}
\end{frame}

\begin{frame}
	\frametitle{Tiefe der Grammatik}
	\begin{columns}
		\column{\dimexpr\paperwidth-10pt}

		\note[item]{WidthFirst produziert auf jedem Datensatz die flachsten Grammatiken}
		\note[item]{Nächsttiefere sind je nach Datensatz ChildArea und Sequitur}
		\note[item]{Die dann nächsttieferen: HeightFirstArea und HeightAdvantageArea \begin{itemize}
			\item Wieder quasi identisch
		\end{itemize}}
		\note[item]{Mit enormen Abstand tiefste Grammatiken von RePair}

		\begin{tikzpicture}
			\begin{axis}[depthplot, title={DNA}, name=dnatime, ylabel={Tiefe}, ymode=log]
			   
                \addplot coordinates { (1,7.0) (2,7.0) (3,9.0) (4,9.0) (5,9.0) };
                \addplot coordinates { (1,48.0) (2,52.0) (3,58.0) (4,58.0) (5,58.0) };
                \addplot coordinates { (1,49.0) (2,51.0) (3,57.0) (4,58.0) (5,57.0) };
                \addplot coordinates { (1,3.0) (2,3.0) (3,3.0) (4,3.0) (5,3.0) };
                \addplot coordinates { (1,335.0) (2,532.0) (3,2403.0) (4,1635.0) (5,3034.0) };
                \addplot coordinates { (1,11.0) (2,11.0) (3,12.0) (4,12.0) (5,12.0) };
				
				\addlegendentry{ChildArea};
				\addlegendentry{HeightAdvantageArea};
				\addlegendentry{HeightFirstArea};
				\addlegendentry{WidthFirstArea};
				\addlegendentry{RePair};
				\addlegendentry{Sequitur};
			\end{axis}
	
			\begin{axis}[depthplot, title={Proteins}, at=(dnatime.east), rightdist, name=proteinstime, sizexlabel, ymode=log]
				\addplot coordinates { (1,36.0) (2,43.0) (3,47.0) (4,49.0) (5,50.0) };
                \addplot coordinates { (1,53.0) (2,59.0) (3,62.0) (4,64.0) (5,64.0) };
                \addplot coordinates { (1,50.0) (2,52.0) (3,62.0) (4,64.0) (5,64.0) };
                \addplot coordinates { (1,3.0) (2,4.0) (3,4.0) (4,4.0) (5,4.0) };
                \addplot coordinates { (1,1628.0) (2,2819.0) (3,3413.0) (4,2982.0) (5,3733.0) };
                \addplot coordinates { (1,20.0) (2,24.0) (3,24.0) (4,24.0) (5,24.0) };
			\end{axis}
			
			\begin{axis}[depthplot, title={XML}, at=(proteinstime.east), rightdist, name=xmltime, ymode=log]
				\addplot coordinates { (1,17.0) (2,16.0) (3,17.0) (4,17.0) (5,16.0) };
                \addplot coordinates { (1,28.0) (2,33.0) (3,36.0) (4,36.0) (5,36.0) };
                \addplot coordinates { (1,26.0) (2,31.0) (3,34.0) (4,33.0) (5,33.0) };
                \addplot coordinates { (1,7.0) (2,8.0) (3,7.0) (4,7.0) (5,7.0) };
                \addplot coordinates { (1,85.0) (2,116.0) (3,180.0) (4,155.0) (5,243.0) };
                \addplot coordinates { (1,19.0) (2,19.0) (3,19.0) (4,19.0) (5,19.0) };
			\end{axis}
	
			\begin{axis}[depthplot, title={English}, at=(xmltime.east), rightdist, ymode=log]
				\addplot coordinates { (1,17.0) (2,14.0) (3,14.0) (4,14.0) (5,20.0) };
                \addplot coordinates { (1,41.0) (2,45.0) (3,63.0) (4,63.0) (5,64.0) };
                \addplot coordinates { (1,41.0) (2,45.0) (3,62.0) (4,63.0) (5,64.0) };
                \addplot coordinates { (1,9.0) (2,10.0) (3,10.0) (4,11.0) (5,11.0) };
                \addplot coordinates { (1,1679.0) (2,3907.0) (3,2516.0) (4,3216.0) (5,13648.0) };
                \addplot coordinates { (1,17.0) (2,20.0) (3,29.0) (4,29.0) (5,29.0) };
			\end{axis}
		\end{tikzpicture}
	  \end{columns}
\end{frame}

\begin{frame}
	\frametitle{Regelanzahl}
	\begin{columns}
		\column{\dimexpr\paperwidth-10pt}

		\note[item]{Grammatiken mit niedrigster Regelanzahl sind ChildArea und WidthFirstArea mit enormem Abstand}
		\note[item]{Nächstgrößere Regelanzahl: Sequitur}
		\note[item]{Über Sequitur variieren Ergebnisse zwischen den Datensätzen}

		\begin{tikzpicture}
			\begin{axis}[countplot, title={DNA}, name=dnatime, ylabel={Regeln}, ymode=log]
			   
                \addplot coordinates { (1,71.0) (2,78.0) (3,154.0) (4,155.0) (5,187.0) };
                \addplot coordinates { (1,545201.0) (2,1.0341e+06) (3,1.51782e+06) (4,1.98592e+06) (5,2.44195e+06) };
                \addplot coordinates { (1,551990.0) (2,1.04767e+06) (3,1.53673e+06) (4,2.01088e+06) (5,2.47137e+06) };
                \addplot coordinates { (1,63.0) (2,69.0) (3,148.0) (4,148.0) (5,178.0) };
                \addplot coordinates { (1,113457.0) (2,319827.0) (3,303421.0) (4,619826.0) (5,666054.0) };
                \addplot coordinates { (1,119842.0) (2,213943.0) (3,311928.0) (4,409847.0) (5,502206.0) };
				
				\addlegendentry{ChildArea};
				\addlegendentry{HeightAdvantageArea};
				\addlegendentry{HeightFirstArea};
				\addlegendentry{WidthFirstArea};
				\addlegendentry{RePair};
				\addlegendentry{Sequitur};
			\end{axis}
	
			\begin{axis}[countplot, title={Proteins}, at=(dnatime.east), rightdist, name=proteinstime, sizexlabel, ymode=log]
				\addplot coordinates { (1,8497.0) (2,8319.0) (3,8036.0) (4,7568.0) (5,10243.0) };
                \addplot coordinates { (1,462337.0) (2,876976.0) (3,1.43681e+06) (4,2.1312e+06) (5,2.80116e+06) };
                \addplot coordinates { (1,481653.0) (2,913402.0) (3,1.50013e+06) (4,2.23379e+06) (5,2.9401e+06) };
                \addplot coordinates { (1,4890.0) (2,4931.0) (3,5214.0) (4,5242.0) (5,6884.0) };
                \addplot coordinates { (1,593074.0) (2,1.3128e+06) (3,2.1662e+06) (4,1.81616e+06) (5,2.02654e+06) };
                \addplot coordinates { (1,219631.0) (2,377264.0) (3,501559.0) (4,624574.0) (5,746617.0) };
			\end{axis}
			
			\begin{axis}[countplot, title={XML}, at=(proteinstime.east), rightdist, name=xmltime, ymode=log]
				\addplot coordinates { (1,24439.0) (2,34744.0) (3,43287.0) (4,49694.0) (5,55222.0) };
                \addplot coordinates { (1,196226.0) (2,361346.0) (3,523543.0) (4,668901.0) (5,815467.0) };
                \addplot coordinates { (1,206883.0) (2,380826.0) (3,552165.0) (4,706238.0) (5,861838.0) };
                \addplot coordinates { (1,19209.0) (2,26166.0) (3,31415.0) (4,34564.0) (5,38301.0) };
                \addplot coordinates { (1,189468.0) (2,393154.0) (3,546340.0) (4,656049.0) (5,800207.0) };
                \addplot coordinates { (1,148736.0) (2,271075.0) (3,387604.0) (4,490670.0) (5,600019.0) };
			\end{axis}
	
			\begin{axis}[countplot, title={English}, at=(xmltime.east), rightdist, ymode=log]
				\addplot coordinates { (1,17499.0) (2,22996.0) (3,29151.0) (4,31880.0) (5,36583.0) };
                \addplot coordinates { (1,352197.0) (2,592325.0) (3,959541.0) (4,1.33699e+06) (5,1.62917e+06) };
                \addplot coordinates { (1,375102.0) (2,630449.0) (3,1.02432e+06) (4,1.4257e+06) (5,1.73823e+06) };
                \addplot coordinates { (1,14226.0) (2,18024.0) (3,22198.0) (4,24030.0) (5,27814.0) };
                \addplot coordinates { (1,1.76936e+06) (2,3.96516e+06) (3,4.78321e+06) (4,5.97792e+06) (5,6.9114e+06) };
                \addplot coordinates { (1,203426.0) (2,352571.0) (3,510494.0) (4,657091.0) (5,783690.0) };
			\end{axis}
		\end{tikzpicture}
	  \end{columns}
\end{frame}

\pgfplotsset{
  log ticks with fixed point,
}

\begin{frame}
	\frametitle{Regel-Durchschnittslänge}

	\note[item]{RePair lohnt nicht zu betrachten, da alle Regeln außer der Startregel sowieso Länge 2 besitzen}
	\note[item]{Sehr unterschiedliche Ergebnisse für die Verschiedenen Datensätze}
	\note[item]{Meistens alle Algorithmen mit Durchschnittslänge zwischen 2 und 4}
	\note[item]{Ausnahmen: \begin{itemize}
		\item ChildArea in Proteins: Durchschnittslänge von bis zu 22
		\item WidthFirst und ChildArea in English: Jeweils 40 und 96 max Durchschnittslänge
	\end{itemize}}


	\begin{columns}
		\column{\dimexpr\paperwidth-10pt}

		\begin{tikzpicture}
			\begin{axis}[lengthplot, title={DNA}, name=dnatime, ylabel={Regellänge}, ymode=log, yticklabel style={
				/pgf/number format/precision=2,
				/pgf/number format/fixed}]
			   
                \addplot coordinates { (1,2.01429) (2,2.02597) (3,2.02614) (4,2.01299) (5,2.04301) };
                \addplot coordinates { (1,2.17989) (2,2.18967) (3,2.18764) (4,2.19015) (5,2.19205) };
                \addplot coordinates { (1,2.19303) (2,2.20367) (3,2.20288) (4,2.20553) (5,2.20552) };
                \addplot coordinates { (1,2.0) (2,2.0) (3,2.01361) (4,2.0) (5,2.0339) };
                \addplot coordinates { (1,2.0) (2,2.0) (3,2.0) (4,2.0) (5,2.0) };
                \addplot coordinates { (1,2.04732) (2,2.07584) (3,2.08106) (4,2.08939) (5,2.09551) };
				
				\addlegendentry{ChildArea};
				\addlegendentry{HeightAdvantageArea};
				\addlegendentry{HeightFirstArea};
				\addlegendentry{WidthFirstArea};
				\addlegendentry{RePair};
				\addlegendentry{Sequitur};
			\end{axis}
	
			\begin{axis}[lengthplot, title={Proteins}, at=(dnatime.east), rightdist, name=proteinstime, sizexlabel, ymode=log, yticklabel style={
				/pgf/number format/precision=2,
				/pgf/number format/fixed}]
				\addplot coordinates { (1,13.4101) (2,15.4264) (3,21.9736) (4,21.8799) (5,22.0033) };
                \addplot coordinates { (1,2.53079) (2,2.67598) (3,2.56286) (4,2.41029) (5,2.35682) };
                \addplot coordinates { (1,2.52303) (2,2.66375) (3,2.55465) (4,2.40679) (5,2.35552) };
                \addplot coordinates { (1,2.11004) (2,2.12049) (3,3.79762) (4,3.95669) (5,2.39271) };
                \addplot coordinates { (1,2.0) (2,2.0) (3,2.0) (4,2.0) (5,2.0) };
                \addplot coordinates { (1,2.96105) (2,3.44833) (3,3.52837) (4,3.27092) (5,3.17545) };
			\end{axis}
			
			\begin{axis}[lengthplot, title={XML}, at=(proteinstime.east), rightdist, name=xmltime, ymode=log]
				\addplot coordinates { (1,2.57026) (2,2.55804) (3,2.59017) (4,2.59298) (5,2.61899) };
                \addplot coordinates { (1,2.19464) (2,2.18694) (3,2.18079) (4,2.18614) (5,2.19015) };
                \addplot coordinates { (1,2.18764) (2,2.18324) (3,2.17724) (4,2.1818) (5,2.18434) };
                \addplot coordinates { (1,2.94409) (2,2.8991) (3,2.88215) (4,2.9592) (5,2.90167) };
                \addplot coordinates { (1,2.0) (2,2.0) (3,2.0) (4,2.0) (5,2.0) };
                \addplot coordinates { (1,2.20564) (2,2.1848) (3,2.18534) (4,2.17352) (5,2.16774) };
			\end{axis}
	
			\begin{axis}[lengthplot, title={English}, at=(xmltime.east), rightdist, ymode=log]
				\addplot coordinates { (1,52.2857) (2,96.1143) (3,85.5238) (4,85.6127) (5,83.0841) };
                \addplot coordinates { (1,3.19531) (2,3.61483) (3,3.21938) (4,2.98099) (5,2.96945) };
                \addplot coordinates { (1,3.137) (2,3.53201) (3,3.15797) (4,2.93691) (5,2.92538) };
                \addplot coordinates { (1,25.7272) (2,40.1565) (3,32.6568) (4,29.4731) (5,27.2751) };
                \addplot coordinates { (1,2.0) (2,2.0) (3,2.0) (4,2.0) (5,2.0) };
                \addplot coordinates { (1,3.73509) (2,4.26732) (3,3.948) (4,3.71414) (5,3.77808) };
			\end{axis}
		\end{tikzpicture}
	  \end{columns}
\end{frame}

\begin{frame}
	\frametitle{Laufzeitanteile}

	\note<1>[item]{Berechnung des Enhanced Suffix Array etwa $10\%$}
	\note<1>[item]{Konstruieren der Prioritätswarteschlange und das Entnehmen der Elemente auch etwa $10\%$}
	\note<1>[item]{Das Sortieren der Positionen der Vorkommen, die ersetzt werden sollen etwa $5\%$ \begin{itemize}
		\item für das Bereinigen der Vorkommen benötigt
	\end{itemize}}
	\note<1>[item]{Eigentliches Substituieren der Vorkommen: $5\%$}
	\note<2->[item]{Bereinigen der Vorkommen vor der Substitution: $57\%$ \begin{itemize}
		\item Mit abstand der Größte Laufzeitanteil
		\item Viele Anfragen an die RuleIntervalIndex Datenstruktur
		\item höchste Priorität für Laufzeitverbesserungen
		\item Wichtig: $24$\% davon stammen aus einer unbekannten Quelle \begin{itemize}
			\item Profiling mit VisualVM
			\item Diese $24\%$ in einem Methodenaufruf nur als Self-Time angezeigt
			\item Self-Time: normalerweise die Zeit, die in einer Methode selbst verbraucht wird, ohne weitere Methodenaufrufe.
			\item Diese Methode besteht aber lediglich aus einem einzigen Methodenaufruf
			\item Bisher kein Problem gefunden. Möglicherweise eine Eigenart der JVM  
		\end{itemize}
	\end{itemize}}
	\note<2->[item]{Überführen in explizite Darstellung: etwa $8\%$ \begin{itemize}
		\item Transformiert RuleIntervalIndex und den Eingabestring in eine Repräsentation in der Regeln explizit gespeichert werden
		\item sonst müsste immer der Originalstring im Speicher gehalten werden.
	\end{itemize}}


	\begin{tikzpicture}
		\begin{axis}[
			ybar stacked, 
			xtick={0}, 
			xticklabels={ },
			width=3cm,
			height=6cm,
			y unit={\%},
			ymin=0,
			ymax=100,
			ylabel={Laufzeitanteil},
			xticklabel style={align=center},
			legend style={at={(1.1,0.5)}, anchor=west},
			cycle list name=algso
			]
			\addplot coordinates { (0,9.4) };
			\addplot coordinates { (0,9.8) };
			\addplot coordinates { (0,5.5) };
			\addplot coordinates { (0,5) };
			\addplot coordinates { (0,57.7) };
			\addplot coordinates { (0,8.1) };
			\addplot coordinates { (0,4.5) };


			\addlegendentry{Enhanced Suffix Array};
			\addlegendentry{Prioritätswarteschlange-Operationen};
			\addlegendentry{Sortieren von $positions$};
			\addlegendentry{Substituieren von Vorkommen};
			\addlegendentry{Bereinigen von $positions$};
			\addlegendentry{Einheitliche Repräsentation};
			\addlegendentry{Rest};
		\end{axis}
	\end{tikzpicture}

\end{frame}

\subsection{Fazit}

\begin{frame}
	\frametitle{Fazit}
	
	\note<1->[item]{Flächenfunktionen mit kleinster Grammatik: \texttt{HeightFirstArea} und \texttt{HeightAdvantageArea} \begin{itemize}
		\item nur etwas mehr Laufzeit, aber Unterschied klein
		\item HeightAdvantageArea etwas kleinere Grammatiken, aber der Unterschied ist sehr gering
	\end{itemize}}
	\note<2->[item]{In der Praxis scheinbar lineare Laufzeit trotz schlechterer Abschätzung}
	\note<3->[item]{Sequitur und RePair schneller und Sequitur produziert die kleinsten Grammatiken}
	\note<4->[item]{Insgesamt \begin{itemize}
		\item viel kleinere Grammatiken, wenn Höhe eines LCP-Intervalls stärker als die Breite priorisiert wird
	\end{itemize}}
	\note<5->[item]{Allerdings allgemeingültiges Fazit besser \begin{itemize}
		\item lässt sich nicht sagen \enquote{Alg. A ist \emph{besser} als Alg. B}
		\item Kodierung muss noch erfolgen und hier könnten andere Eigenschaften als nur Größe wichtig sein
	\end{itemize}}


	\begin{itemize}
		\item<1-> AreaComp Grammatiken in Größe zum Teil konkurrenzfähig mit Sequitur Grammatiken
		\item<1-> Flächenfunktionen mit kleinster Grammatik: \texttt{HeightFirstArea} und \texttt{HeightAdvantageArea}
		\item<2-> In der Praxis scheinbar lineare Laufzeit
		\item<3-> Trotzdem langsamer als RePair und Sequitur \begin{itemize}
			\item Letzterer mit kleinsten Grammatiken und bester Laufzeit
		\end{itemize} 
		\item<4-> Insgesamt: kleinere Grammatiken, wenn Höhe eines LCP-Intervalls priorisiert wird
		\item<5-> Allerdings kein allgemeingültige Bewertung möglich
	\end{itemize}

\end{frame}