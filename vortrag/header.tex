% header.tex
\documentclass[notes]{beamer}
\usepackage{pgfpages}
\setbeamertemplate{note page}[plain]
\setbeameroption{show notes on second screen=right}
%\usepackage[a4paper,left=3.5cm,right=2.5cm,bottom=3.5cm,top=3cm]{geometry}

\usepackage[ngerman,english]{babel}

%\usepackage[pdftex]{graphicx, color}
\usepackage{amsmath,amssymb}
%\usepackage{amsthm}

\usepackage{hyperref}
\usepackage{float}

% Korrekte Darstellung der Umlaute
\usepackage[utf8]{inputenc}
\usepackage[T1]{fontenc}

% Algorithmen
%\usepackage[ruled,chapter]{algorithm}
%\usepackage{algorithmicx}
\usepackage[ngerman, ruled, algosection, linesnumbered, noend]{algorithm2e}

\usepackage{enumerate}
%\usepackage{enumitem}
\usepackage{multicol}

% table cell colors
\usepackage{xcolor}
%\newcommand{\cc}[1]{\cellcolor{#1}}

% Bibtex deutsch
\usepackage[backend=biber,sortlocale=de_DE,style=numeric]{biblatex}
\addbibresource{literatur.bib}

% URLs
\usepackage{url}

% subfig
\usepackage{subfig}

% TikZ
\usepackage{tikz}
\usetikzlibrary{shapes, positioning, calc, matrix, arrows.meta, decorations.text, fit, external}
%\tikzexternalize

\usepackage{pgfplots}
\pgfplotsset{compat=1.18}
\usepgfplotslibrary{units}

% csquotes
\usepackage{csquotes}

% Caption Packet
\usepackage[margin=0pt,font=small,labelfont=bf]{caption}
%\usepackage{subcaption}
% Gliederung einstellen
%\setcounter{secnumdepth}{5}
%\setcounter{tocdepth}{5}
\hfuzz=100.002pt 

% Algorithmen anpassen %
%\renewcommand{\algorithmicrequire}{\textit{Eingabe:}}
%\renewcommand{\algorithmicensure}{\textit{Ausgabe:}}
\floatname{algorithm}{Algorithmus}
%\renewcommand{\listalgorithmname}{Algorithmenverzeichnis}
%\renewcommand{\algorithmiccomment}[1]{\color{grau}{// #1}}
\SetKw{KwNull}{null}
\SetKw{KwTrue}{true}
\SetKw{KwFalse}{false}
\SetKw{And}{and}
\SetKw{Or}{or}
\SetKw{Not}{not}
\SetKw{KwBreak}{break}
\SetKw{KwContinue}{continue}
\SetKwRepeat{Do}{do}{while}


% Abkuerzungen richtig formatieren %
\usepackage{xspace}
\newcommand{\vgl}{vgl.\@\xspace} 
\newcommand{\zB}{z.\nolinebreak[4]\hspace{0.125em}\nolinebreak[4]B.\@\xspace}
\newcommand{\bzw}{bzw.\@\xspace}
\newcommand{\dahe}{d.\nolinebreak[4]\hspace{0.125em}h.\nolinebreak[4]\@\xspace}
\newcommand{\etc}{etc.\@\xspace}
\newcommand{\evtl}{evtl.\@\xspace}
\newcommand{\ggf}{ggf.\@\xspace}
\newcommand{\bzgl}{bzgl.\@\xspace}
\newcommand{\so}{s.\nolinebreak[4]\hspace{0.125em}\nolinebreak[4]o.\@\xspace}
\newcommand{\iA}{i.\nolinebreak[4]\hspace{0.125em}\nolinebreak[4]A.\@\xspace}
\newcommand{\sa}{s.\nolinebreak[4]\hspace{0.125em}\nolinebreak[4]a.\@\xspace}
\newcommand{\su}{s.\nolinebreak[4]\hspace{0.125em}\nolinebreak[4]u.\@\xspace}
\newcommand{\ua}{u.\nolinebreak[4]\hspace{0.125em}\nolinebreak[4]a.\@\xspace}
\newcommand{\og}{o.\nolinebreak[4]\hspace{0.125em}\nolinebreak[4]g.\@\xspace}
\newcommand{\oBdA}{o.\nolinebreak[4]\hspace{0.125em}\nolinebreak[4]B.\nolinebreak[4]\hspace{0.125em}d.\nolinebreak[4]\hspace{0.125em}A.\@\xspace}
\newcommand{\OBdA}{O.\nolinebreak[4]\hspace{0.125em}\nolinebreak[4]B.\nolinebreak[4]\hspace{0.125em}d.\nolinebreak[4]\hspace{0.125em}A.\@\xspace}

% Leere Seite ohne Seitennummer, nächste Seite rechts
\newcommand{\blankpage}{
 \clearpage{\pagestyle{empty}\cleardoublepage}
}

\makeatletter
\newcommand{\xRightarrow}[2][]{\ext@arrow 0359\Rightarrowfill@{#1}{#2}}
\makeatother

\makeatletter
\def\beamer@setupnote{%
  \gdef\beamer@notesactions{%
    \beamer@outsideframenote{%
      \beamer@atbeginnote%
      \beamer@notes%
      \ifx\beamer@noteitems\@empty\else
      \begin{itemize}\itemsep=0pt\parskip=0pt%
        \beamer@noteitems%
      \end{itemize}%
      \fi%
      \beamer@atendnote%
    }%
    \gdef\beamer@notesactions{}%
  }
}
\makeatother

% EOF

%\usepackage{syntonly}
% syntax check only?
%\syntaxonly

\setcounter{tocdepth}{1}