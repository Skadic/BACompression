% !TeX root = ../../main.tex


\tikzset{
    table nodes/.style={
        rectangle,
        draw=black,
        align=center,
        minimum height=7mm,
        text depth=0.5ex,
        text height=2ex,
        inner xsep=0pt,
        outer sep=0pt
    },      
    table/.style={
        matrix of nodes,
        row sep=-\pgflinewidth,
        column sep=-\pgflinewidth,
        nodes={
            table nodes
        },
        execute at empty cell={\node[draw=none]{};}
    }
}


\section{AreaComp V4}
\label{v4}

Dies ist bis jetzt die finale Version dieses Algorithmus. 

\subsection{Bessere Predecessor-Datenstruktur}

Bisher verwendete \texttt{RuleIntervalIndex} Red-Black-Trees als zugrundeliegende Predecessor-Datenstruktur. Diese benötigen $\mathcal{O}(\log n)$ Laufzeit für jede Operation. Da aber etwa die $\texttt{get}$ und $\texttt{predecessor}$ Operation sehr oft aufgerufen werden, stellt dies ein Bottleneck für die Laufzeit dar.
Die Idee ist nun, Red-Black-Trees durch eine für diese Zwecke effizientere Datenstruktur zu ersetzen.

Dazu bietet sich die in Kapitel 4 von \cite{dinklage_engineering_2021} von Dinklage et al. beschriebene Predecessor-Datenstruktur an. Im weiteren Verlauf nennen wir diese \texttt{BucketPred}. Im Code für AreaComp ist sie unter demselben Namen zu finden. Wir nutzen allerdings hier eine assoziative Variante dieser Datenstruktur. Die ursprüngliche \texttt{BucketPred} Datenstruktur speichert nur die ganzzahligen Schlüssel. Wir erweitern \texttt{BucketPred} durch eine Hashtabelle. Diese bildet die in der Predecessor-Datenstruktur gespeicherten Schlüssel auf zugehörige Daten ab. Es gibt nur eine Hashtabelle für die gesamte Datenstruktur.   
Eine lokale Hashtabelle für jeden Bucket, scheint bei Tests keine Vorteile zu bieten.\\
Durch diese Hashtabelle erhalten wir $\mathcal{O}(1)$ erwartete Laufzeit für $\texttt{get}$ Operationen, $\mathcal{O}(b)$ Worst-Case Laufzeit für $\texttt{predecessor}$ Operationen und $\mathcal{O}(\tfrac{u}{b})$ Worst-Case Laufzeit für $\texttt{insert}$ Operationen. (siehe \autoref{bucketpredds}) 

Wie in \cite{dinklage_engineering_2021} beschrieben, ist es ebenfalls möglich, das hier verwendete Array von Buckets durch eine Hashtabelle zu ersetzen. Dies würde für $\texttt{insert}$ Operationen eine erwartete $\mathcal{O}(1)$ Laufzeit bedeuten, da nun nicht mehr im Array die $\mathcal{O}(\tfrac{u}{b})$ Pointer verändert werden müssen. Allerdings würde dies auch eine Laufzeit von $\mathcal{O}(\tfrac{u}{b})$ für $\texttt{predecessor}$ Operationen bedeuten, da potenziell $\mathcal{O}(\tfrac{u}{b})$ Anfragen an die Hashtabelle gestellt werden müssen, um einen aktiven Bucket zu finden. 

Allerdings wird $\texttt{predecessor}$ viel häufiger benötigt als $\texttt{insert}$ Operationen. Daher ist es von größerer Bedeutung, dass $\texttt{predecessor}$ Aufrufe performant sind. Daher eignet sich hier das Bucket-Array besser.

%Da $u = |s|$ durch den Eingabestring $s$ gegeben ist, ist also die Wahl von $b = 2^k$, für ein $k \in \mathbb{N}$, interessant. Dieser widmen wir uns bei der Evaluation.

\subsection{Verbesserung von \texttt{RuleIntervalIndex}}

Wie bereits beschrieben, hat die vorherige Version von \texttt{RuleIntervalIndex} das Problem, dass bestimmte Fälle, in denen eine Substitution legal ist, nicht korrekt erkannt werden. Außerdem ist das Markieren von Bereichen noch ineffizient.  

Zu diesem Zweck verändern wir die Funktionsweise von \texttt{RuleIntervalIndex} etwas. In V3 wurden einzelne Teile von Intervallen gespeichert, sodass immer nur das tiefste Intervall an einem Index gespeichert wird.
Stattdessen speichern wir jetzt alle existierenden Intervalle ganz. Dies tun wir auf eine Art und Weise, die es uns trotzdem ermöglicht, effizient das tiefste verschachtelte Intervall an einem Index zu finden, aber auch gegebenfalls weniger tief verschachtelte Intervalle. 

\subsubsection{Beispiel}

Ein Beispiel für die neue Version der Datenstruktur für den String $abcdeabcd$ und eine gegebene Grammatik ist in \autoref{newrii} abgebildet.

\begin{figure}
    \centering
    \subfloat[Grammatik für $abcdeabcd$]{
        \parbox[c]{5cm}{
            \centering
            $\begin{aligned}
                R_0 &\rightarrow R_1 c d e R_1 c d\\
                R_1 &\rightarrow a b\\
            \end{aligned}$
        }
        \label{newriigrammar}
    }
    \quad
    \subfloat[Die neue \texttt{RuleIntervalIndex} Struktur]{
        \begin{tikzpicture}
        
            \matrix (A) [table, text width=7mm, ampersand replacement=\&] {
                |[draw=none]| $0$ \& |[draw=none]| $1$ \& |[draw=none]| $2$ \& |[draw=none]| $3$ \& |[draw=none]| $4$ \& |[draw=none]| $5$ \& |[draw=none]| $6$ \& |[draw=none]| $7$ \& |[draw=none]| $8$\\
                $a$ \& $b$ \& $c$ \& $d$ \& $e$ \& $a$ \& $b$ \& $c$ \& $d$\\
                \\
                |[draw=none]| \& |[draw=none]|  \&  \&  \&  \& |[draw=none]|  \& |[draw=none]|  \&  \& |[draw=none]|\\
                |[draw=none]| \& |[draw=none]|  \&  \&  \&  \& |[draw=none]|  \& |[draw=none]|  \&  \& |[draw=none]|\\
            };
            
            
            \node[draw, fit=(A-4-1)(A-4-9), table nodes] {$R_0$-$[0..8]$};
            \node[draw, fit=(A-5-1)(A-5-2), table nodes] {$R_1$-$[0..1]$};
            \node[draw, fit=(A-5-6)(A-5-7), table nodes] {$R_1$-$[5..6]$};
        \end{tikzpicture}
    }

    \caption{Die neue \texttt{RuleIntervalIndex} Struktur für den String $abcdeabcd$ und die in \autoref{newriigrammar} abgebildete Grammatik.}
    \label{newrii}
\end{figure}

\subsubsection{Neue \texttt{RuleInterval}s}

In V3 waren \texttt{RuleInterval}s Teile von Ersetzungsintervallen. In V4 ist ein \texttt{RuleInterval} ein \emph{ganzes} Ersetzungsintervall. Demnach bezeichnen wir im Folgenden mit $R_k$-$[i..j]$ nicht mehr den \emph{Teil} eines Ersetzungsintervalls, sondern ein \emph{ganzes} Ersetzungsintervall.

Die Attribute $\texttt{ruleId}$, $\texttt{start}$ und $\texttt{end}$ sind genau wie in \autoref{riiv3}. Die anderen Attribute werden durch die Folgenden ersetzt:

\begin{itemize}[leftmargin=3cm]
    \item[\texttt{parent}] Dies ist ein Pointer auf auf das am tiefsten verschachtelte Ersetzungsintervall, das dieses Intervall umschließt. Dabei muss das $\texttt{parent}$ Intervall nicht unbedingt an demselben Startindex beginnen, wie das Intervall selbst. In \autoref{newrii} zeigt der \texttt{parent}-Pointer der beiden $R_1$ Intervalle auf das $R_0$ Intervall.
    \item[\texttt{firstAtStartIndex}] Ein Pointer auf das am \emph{wenigsten} verschachtelte Ersetzungsintervall, das an demselben Startindex beginnt, wie das Intervall selbst. Im Fall des Beispiels zeigen die \texttt{firstAtStartIndex} aller Intervalle jeweils auf sich selbst, da keine zwei Intervalle an demselben Index beginnen.
    \item[\texttt{nextAtStartIndex}] Ein Pointer auf das Ersetzungsintervall, das am selben Startindex beginnt und das nächst-tiefer verschachtelt ist
\end{itemize}

Die erste Idee war, in der zugrundeliegenden Predecessor-Struktur dynamische Arrays von $\texttt{RuleInterval}$s zu speichern. Dies war allerdings aufgrund des Speicherverbrauchs und des Laufzeit-Overhead von Allokationen und Reallokationen unhaltbar. Als Verbesserung sind die Elemente der Predecessor Struktur nun für jeden Index das jeweils das am tiefsten verschachtelte Intervall, das an diesem Startindex beginnt. Auf die anderen Intervalle, die an demselben Startindex beginnen, kann mithilfe der $\texttt{firstAtStartIndex}$, $\texttt{nextAtStartIndex}$ und $\texttt{parent}$-Pointer zugegriffen werden.

\subsubsection{Bestimmen der umschließenden Intervalle}

Wir können nun bestimmen, welches das tiefste verschachtelte Ersetzungsintervall ist, das ein Intervall $[from..to]$ umschließt.
Dies ist mithilfe von \autoref{intervalcontainingv4} möglich.

\begin{algorithm}[t]
    \KwIn {$from, to$ Start- und Endindizes}
    \KwOut{Das tiefste Ersetzungsintervall, das sowohl $from$ als auch $to$ umschließt}
    $current \leftarrow floorInterval(from)$\tcp*{\texttt{predecessor}-Anfrage um das Vorgänger-Intervall zu bestimmen}
    \While {$current \neq$ \KwNull} {
        \If {$current.start \leq from \textbf{ and } to \leq current.end$} {
            \KwRet{$current$}
        }
        \eIf {$current.firstAtStartIndex \leq from \textbf{ and } to \leq current.firstAtStartIndex.end$} {
            $current \leftarrow current.parent$\;
        }{
            $current \leftarrow current.firstAtStartIndex$\;
            \If {$current.parent \neq$ \KwNull} {
                $current \leftarrow current.parent$ 
            }
        }
    }
    \KwRet{\KwNull}
    \caption{intervalContaining}
    \label{intervalcontainingv4}
\end{algorithm}

Zuerst bestimmen wir mithilfe der zugrundeliegenden Predecessor Struktur das tiefste verschachtelte Vorgänger-Intervall des Startingindex $from$ (Zeile $1$). 
In der While-Schleife wird nun geprüft, ob das jetzige Intervall auch beide Grenzen einschließt. Ist dies der Fall, so haben wir das tiefste Intervall, das das Intervall $[from..to]$ einschießt, gefunden.\\
Ist dies nicht der Fall, so wird nacheinander entlang der $\texttt{parent}$-Pointer iteriert, bis ein Intervall gefunden wird, das beide Grenzen enthält. Da das Intervall der Startregel $R_0$ die ganze Länge des Eingabestring einnimmt, ist es also garantiert, dass solch ein Intervall auch existiert. Diese Methode kann auch das tiefste Intervall an einem Index $i$ bestimmen, indem $\texttt{intervalContaining(i, i)}$ aufgerufen wird.

In dem Fall, dass selbst das am wenigsten verschachtelte Intervall an diesem Startindex (\texttt{firstAtStartIndex}) nicht die gegebenen Grenzen umschließt, so kann direkt zum Eltern-Intervall von \texttt{firstAtStartIndex} gesprungen werden, da somit keines der übersprungenen Intervalle die Grenzen umschließt.

Die dominierenden Teile der Laufzeit von \autoref{intervalcontainingv4} bestehen aus dem $floorInterval$-Aufruf mit $\mathcal{O}(b)$ Laufzeit, wobei $b$ die Größe der Buckets von \texttt{BucketPred} ist. Dazu kommt die While-Schleife. Ist $d$ die Tiefe der Grammatik wenn AreaComp terminiert, so können maximal $d$ Wiederholungen stattfinden, da durch die \texttt{parent}-Pointer iteriert wird. Insgesamt folgt eine Laufzeit von $\mathcal{O}(b + d)$. 

\subsubsection{Markieren von Ersetzungsintervallen}

\begin{algorithm}[t]
    \KwIn{$id:$ ID der Regel, $start, end:$ Start- und Endindex des Intervalls, $pred:$ Die \texttt{BucketPred} Datenstruktur}
    $current \leftarrow$ tiefstes Intervall mit Startindex $start$\;
    $interval \leftarrow R_{id}$-$[start..end]$\;

    \If{$current =$ \KwNull} {
        $parent \leftarrow intervalContaining(start, end)$\;
        Setze Pointer von $interval$\;
        $pred.insert(start, interval)$\;
    }
    \ElseIf{$interval \subset current$} {
        Setze Pointer von $interval$\tcp*{Dann ist $current$ das Elternintervall von $interval$}
        $pred.insert(start, interval)$\;
    }
    \Else {
        $current \leftarrow$ Iteriere solange durch $\texttt{parent}$-Pointer von $interval$, bis das Elternintervall von $current$ $interval$ enthält\;
        Setze Pointer von $interval$\;
    }
    Iteriere durch alle \texttt{RuleInterval}s, die Teilintervalle von $[start..end]$ sind und setze ihre $\texttt{parent}$-Pointer auf $interval$, falls nötig\;
    \caption{mark}
    \label{markv4}
\end{algorithm}

In \autoref{markv4} ist der Algorithmus zum Markieren von Ersetzungsintervallen beschrieben.
Hier  wird ein neues $\texttt{RuleInterval}$ $R_i$-$[start..end]$ in die Datenstruktur eingefügt. 
Zunächst wird geprüft, ob es bereits ein Intervall gibt, das bei Startindex $start$ beginnt.
Hier muss zwischen verschiedenen Fällen unterschieden werden.

\begin{itemize}[leftmargin=1.5cm]
    \item[\textbf{Fall 1}] Falls kein solches Intervall existiert, so wird mithilfe von\\
    $\texttt{intervalContaining(start, end)}$ das Eltern-Intervall des neuen Intervalls bestimmt und der $\texttt{parent}$-Pointer entsprechend gesetzt.  Es wird nun nur noch das Intervall in die Predecessor Datenstruktur eingefügt.
    \item[\textbf{Fall 2}] Im Fall, dass ein solches Intervall existiert, kann es sein, dass $R_i$-$[start..end]$ das nun am tiefsten verschachtelte Intervall ist, das an diesem Startindex beginnt. Ist das der Fall, so werden die Pointer des entsprechenden Index gesetzt und das neue Intervall ersetzt nun das vorherige in der Predecessor Datenstruktur, da es nun das tiefste Intervall an diesem Index ist.\\
    Ist das Intervall nicht das tiefste an diesem Index, so kann durch die $\texttt{parent}$-Pointer iteriert werden, bis ein Intervall an diesem Startindex gefunden wird, das das neue Intervall enthält. An diese Stelle muss dann das neue Intervall eingefügt werden. Es kann nun sein, das es kein solches Intervall gibt. Dann ist das neue Intervall, das größte und damit das am wenigsten verschachtelte Intervall, das an diesem Startindex beginnt. Ist das der Fall, so muss der $\texttt{firstAtStartIndex}$ aller Intervalle an diesem Startindex, auf das neue Intervall gesetzt werden. Dies kann während der ohnehin schon stattfindenden Iteration geschehen. 
\end{itemize} 

Es kann nun sein, dass durch das Einfügen eines neuen Intervalls die $\texttt{parent}$-Pointer anderer Intervalle fehlerhaft werden, da sich das neue Intervall zwischen das Eltern- und
Kindintervall schieben. Dies wird nun dadurch behoben, dass über alle Intervalle der Datenstruktur, die im Intervall $[start+1..end]$ liegen, iteriert wird. In dieser Iteration wird der \texttt{parent}-Pointer auf das neue Intervall gesetzt. 
\label{v4marking}

\paragraph{Laufzeit}

Sei $k := end - start + 1$ also die Länge des Musters, für das wir das Intervall markieren und $id$ die ID der Regel. Das neue Intervall ist also $R_{id}$-$[start..end]$.

In jedem Fall muss bestimmt werden, ob es ein Intervall in der Datenstruktur gibt, dass bei Index $start$ beginnt. Dies benötigt erwartete $\mathcal{O}(1)$ Laufzeit aufgrund der Hashtabelle der \texttt{BucketPred} Datenstruktur. 

Im Fall, dass kein solches Intervall existiert, muss das tiefste verschachtelte Intervall bestimmt werden, dass $[start..end]$ enthält. Dies kann mit \texttt{BucketPred} in $\mathcal{O}(b + d)$ bestimmt werden, wobei $b$ die Bucketgröße und $d$ die Tiefe der Grammatik ist, wenn AreaComp terminiert. Das Setzen der Pointer benötigt ebenfalls $\mathcal{O}(1)$ Laufzeit. 
Dann muss $R_{id}$-$[start..end]$ in \texttt{BucketPred} eingefügt werden, was eine erwartete Laufzeit von $\mathcal{O}(1)$ hat, wenn die Datenstruktur nicht sehr spärlich besetzt ist. Zuletzt muss noch durch die Teilintervalle von $[start..end]$ iteriert werden, um ihre \texttt{parent}-Pointer anzupassen. Es können nur $\mathcal{O}(k)$ viele von diesen Intervallen existieren. Da das bestimmen des Nachfolgers eines Intervalls $\mathcal{O}(b)$ Laufzeit benötigt, folgt hierfür eine Laufzeit von $\mathcal{O}(k \cdot b)$, was die Laufzeit dieses Falls dominiert.

Falls ein anderes Intervall $[start..end']$ bereits existiert mit $end < end'$, dann müssen nur die $\texttt{parent}$- und $\texttt{firstAtStartIndex}$-Pointer von $R_{id}$-$[start..end]$ gesetzt werden und es in \texttt{BucketPred} eingefügt werden. Wieder muss durch die Teilintervalle von $[start..end]$ iteriert werden und es folgt wieder eine Laufzeit von $\mathcal{O}(k \cdot b)$.

Falls aber ein anderes Intervall $[start..end']$ bereits existiert mit $end > end'$, dann ist $R_{id}$-$[start..end]$ nicht das tiefste Intervall an diesem Index. Dann wird eine Laufzeit von $\mathcal{O}(d)$ benötigt, um die Stelle in der verketteten Struktur zu finden, an die $R_{id}$-$[start..end]$ eingefügt werden muss. Dann muss wieder durch die Teilintervalle iteriert werden.
Es folgt eine Gesamtlaufzeit von ${O}(k \cdot b + d)$.