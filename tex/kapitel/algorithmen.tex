\chapter{Bekannte Grammatikkompressionsalgorithmen}
\label{algorithmen}

Grammatikbasierte Textkompression wurde schon mehrmals erforscht, etwa in \cite{nevill-manning_identifying_1997, larsson_offline_1999, apostolico_off-line_2000}. Um einen Vergleich mit anderen Algorithmen zu bieten, werden hier die Algorithmen Sequitur \cite{nevill-manning_identifying_1997} und Re-Pair \cite{larsson_offline_1999} vorgestellt.

\section{Sequitur}
\label{sequitur}

Sequitur \cite{nevill-manning_identifying_1997} ist ein Grammatikkompressions-Algorithmus, der On-line arbeitet. Das heißt, dass der Eingabestring nicht vollständig zu Anfang des Algorithmus im Speicher vorhanden sein muss, sondern auch stückweise eingelesen werden kann. Der Eingabestring wird zeichenweise eingelesen und wiederholt vorkommende Zeichenpaare (Digramme) durch Nichtterminale ersetzt.

\subsection{Eigenschaften der erzeugten Grammatik}

Der Algorithmus berechnet eine Grammatik, dessen Produktionsregeln folgende zwei Eigenschaften besitzen:
\begin{enumerate}
	\item[$p_1$] \emph{Digram-Uniqueness}:\\
	Kein Paar von aufeinanderfolgenden Zeichen kommt in der Grammatik öfter als einmal vor.
	\item[$p_2$] \emph{Rule-Utility}:\\
	Jede Produktionsregel der Grammatik wird mehr als einmal verwendet.
\end{enumerate}

\subsubsection{Digram-Uniqueness}

Wird ein neues Symbol gelesen, so wird es an die rechte Seite der Startregel $S$ angehängt. Dieses Symbol formt mit seinem Vorgänger ein neues Digramm. Kommt dieses Digramm an anderer Stelle in der Grammatik vor, so ist Digram-Uniqueness verletzt. 
Existiert bereits eine Regel in der Grammatik, die auf dieses Digramm abbildet, wird dieses Vorkommen durch das Nichtterminal dieser Regel ersetzt. 
Falls nicht, erzeuge eine neue Regel mit einem neuen Nichtterminal, die dieses Digramm erzeugt. Ersetze dann die ursprünglichen zwei Vorkommen durch das neue Nichtterminal.

\subsubsection{Rule-Utility}

Es kann vorkommen, dass durch Substitutionen ein Nichtterminal nur noch einmal in der Grammatik vorkommt. In diesem Fall ist Rule-Utility verletzt. Ersetze nun dieses eine Vorkommen des Nichtterminals durch die rechte Seite der zugehörigen Produktionsregel, und lösche diese. So können auch Produktionsregeln mit rechten Seiten länger als 2 entstehen.

\subsection{Implementation}

Der Algorithmus arbeitet durch das Aufrechterhalten von Digram-Uniqueness und Rule-Utility. Verletzungen dieser Eigenschaften müssen effizient erkannt werden.

Die Optimale Datenstruktur für zu diesem Zweck muss folgende Operationen effizient ermöglichen:

\begin{itemize}
	\item Ein Symbol an $S$ anhängen
	\item Ein Vorkommen der rechten Seite einer bereits existierenden Regel mit dem zugehörigen Nichtterminal substituieren
	\item Eine neue Regel erzeugen
	\item Eine Regel löschen
\end{itemize}

\subsubsection{Datenstruktur für Regeln und Index}

Sequitur nutzt eine Art doppelt verkettete Ringstruktur zur Implementation von Produktionsregeln. 
Jede Regel besitzt ein sogenanntes \emph{Guard}-Symbol, das den Startpunkt, beziehungsweise den Endpunkt, der Ringstruktur darstellt. Der Nachfolger des Guard-Symbols ist also das erste Symbol der rechten Seite einer Regel und der Vorgänger ist dem entsprechend das letzte Symbol.
Zudem besitzt jedes Vorkommen eines Nichtterminal eine Referenz auf das Guard Symbol der Produktionsregel, die das Nichtterminal repräsentiert. 

\autoref{seqLinkedDataStructure} zeigt ein Beispiel für solch eine Datenstruktur.


\begin{figure}[H]
	\centering

    \subfloat[6cm][Eine Grammatik für den String $aabcdabe$]{%
        \makebox[4cm]{\raisebox{1cm} {
            $\begin{aligned}
                S &\rightarrow aAcdAe\\
                A &\rightarrow ab
            \end{aligned}$
        }}
    }
    \quad
	\subfloat[][Die zugehörige Sequitur-Datenstruktur]{
        \begin{tikzpicture}[
            scale=.9, transform shape,
            normalnode/.style={rectangle, draw=black!60, fill=red!5, thick, minimum size=5mm},
            guardnode/.style={diamond, draw=black!60, fill=red!5, thick, minimum size=3mm},
            ]
            \tikzstyle{every node} = [circle, minimum size=7mm]
            \node[normalnode] (S) at (0, 0) {S};
            \node[guardnode] (G1) at (1, 0) {};
            \node[normalnode] (a1) at (2, 0) {a};
            \node[normalnode] (A1) at (3, 0) {A};
            \node[normalnode] (c) at (4, 0) {c};
            \node[normalnode] (d) at (5, 0) {d};
            \node[normalnode] (A2) at (6, 0) {A};
            \node[normalnode] (e) at (7, 0) {e};
            
            \node[normalnode] (A) at (3, -2) {A};
            \node[guardnode] (G2) at (4, -2) {};
            \node[normalnode] (a2) at (5, -2) {a};
            \node[normalnode] (b1) at (6, -2) {b};
            
            \foreach \from/\to in {G1/a1, a1/A1, A1/c, c/d, d/A2, A2/e, G2/a2, a2/b1}
                \draw [ <-> ] (\from) -- (\to);
    
                
            \draw [ -> ] (S) -- (G1);
            \draw [ <-> ] (G1) to[out=90, in=90, looseness=0.5] (e);
            
            \draw [ -> ] (A) -- (G2);
            \draw [ -> ] (A1) to[out=-90, in=90] (G2);
            \draw [ -> ] (A2) to[out=-90, in=90] (G2);
            \draw [ <-> ] (G2) to[out=45, in=90] (b1);
        \end{tikzpicture}
    }
	\caption{Hier sehen wir die Sequitur-Datenstruktur für die beiden Regeln der dargestellten Grammatik. Rautenförmig dargestellt sind dabei die \emph{Guard}-Symbole der jeweiligen Produktionsregeln. Rechts daran angehängt sind die Symbole der Regel in der doppelt verketteten Ringstruktur. Die Nichtterminale besitzen einen Pointer auf den Guard-Knoten der zugehörigen Regel.}
	\label{seqLinkedDataStructure}
\end{figure}

Für Digram-Uniqueness wird eine Index-Struktur angewandt, um zu überprüfen, ob ein Digramm bereits in der Grammatik existiert. Diese wird als Hashtabelle implementiert, um solche Anfragen in erwarteter konstanter Laufzeit zu bewältigen. Zudem können ebenfalls in erwarteter konstanter Laufzeit Einträge hinzugefügt und auch entfernt werden. Diese Hashtabelle enthält eine Referenz zu dem einzigen Vorkommen jedes Digramms in der Grammatik. 


\subsection{Beispiel}

Wir wenden den Algorithmus beispielhaft auf den String \enquote{$abacabab$} an. Die zu diesem Beispiel zugehörigen Abbildungen sind in \autoref{sequiturexample} zu sehen.\\
Zuerst generiere eine leere Startregel $S$. Dieser Anfangszustand ist in \autoref{seqex1} abgebildet.

\begin{figure}
	\centering
	\subfloat[Die Anfangsdatenstruktur. Die Startregel $S$ ist noch leer.]{
        \parbox[c]{5cm}{
            \centering
            \begin{tikzpicture}[
                transform shape,
                normalnode/.style={rectangle, draw=black!60, thick, minimum size=5mm},
                guardnode/.style={diamond, draw=black!60, thick, minimum size=3mm},
                ]
                \tikzstyle{every node} = [circle, minimum size=7mm]
                \node[normalnode] (S) at (0, 0) {S};
                \node[guardnode] (G1) at (1, 0) {};
                
                \draw [ -> ] (S) -- (G1);
                \draw [ <-> ] (G1) to[out=0, in=90, looseness=5] (G1);
            \end{tikzpicture}
        }
        \label{seqex1}
    }
    \quad
    \subfloat[Die Datenstruktur nach dem Einlesen von $a$, $b$, $a$, $c$, $a$ und $b$.]{
        \begin{tikzpicture}[
            scale=.9, transform shape,
            normalnode/.style={rectangle, draw=black!60, thick, minimum size=5mm},
            guardnode/.style={diamond, draw=black!60, thick, minimum size=3mm},
            ]
            \tikzstyle{every node} = [circle, minimum size=7mm]
            \node[normalnode] (S) at (0, 0) {S};
            \node[guardnode] (G1) at (1, 0) {};
            \node[normalnode, draw=red!60] (a1) at (2, 0) {a};
            \node[normalnode, draw=red!60] (b1) at (3, 0) {b};
            \node[normalnode] (a2) at (4, 0) {a};
            \node[normalnode] (c1) at (5, 0) {c};
            \node[normalnode, draw=red!60] (a3) at (6, 0) {a};
            \node[normalnode, draw=red!60] (b2) at (7, 0) {b};
            
            \draw [ -> ]  (S)  -- (G1);
            
            \foreach \from/\to in {G1/a1, a1/b1, b1/a2, a2/c1, c1/a3, a3/b2}
                \draw [ <-> ] (\from) -- (\to);
                
            \draw [ <-> ] (b2) to[out=90, in=90, looseness=0.5] (G1);
        \end{tikzpicture}
        \label{seqex2}
    }

    \subfloat[$ab$ wurde durch das Nichtterminal $A$ ersetzt. Die Nichtterminale $A$ besitzen einen Pointer auf das Guard-Symbol der Regel $A$.]{
        \begin{tikzpicture}[
            scale=.9, transform shape,
            normalnode/.style={rectangle, draw=black!60, thick, minimum size=5mm},
            guardnode/.style={diamond, draw=black!60, thick, minimum size=3mm},
            ]
            \tikzstyle{every node} = [circle, minimum size=7mm]
            
            % Rule S
            
            \node[normalnode] (S) at (0, 0) {S};
            \node[guardnode] (G1) at (1, 0) {};
            \node[normalnode, draw=red!60] (A11) at (2, 0) {A};
            \node[normalnode, draw=red!60] (a11) at (3, 0) {a};
            \node[normalnode] (c1) at (4, 0) {c};
            \node[normalnode, draw=red!60] (A12) at (5, 0) {A};
            \node[normalnode, draw=red!60] (a12) at (6, 0) {a};
            
            \draw [ -> ]  (S)  -- (G1);
            
            \foreach \from/\to in {G1/A11, A11/a11, a11/c1, c1/A12, A12/a12}
            \draw [ <-> ] (\from) -- (\to);
            
            \draw [ <-> ] (a12) to[out=90, in=90, looseness=0.5] (G1);
            
            % Rule A
            
            \node[normalnode] (A) at (2, -1.5) {A};
            \node[guardnode] (G2) at (3, -1.5) {};
            \node[normalnode] (a2) at (4, -1.5) {a};
            \node[normalnode] (b) at (5, -1.5) {b};
            
            \draw [ -> ]  (A)  -- (G2);
            
            \foreach \from/\to in {G2/a2, a2/b}
            \draw [ <-> ] (\from) -- (\to);
            
            \draw [ <-> ] (b) to[out=90, in=45, looseness=1] (G2);
            
            % Connect S and A
            
            \draw [ -> ] (A1) to[out=-90, in=90] (G2);
            \draw [ -> ] (A2) to[out=-90, in=90] (G2);
        \end{tikzpicture}
        \label{seqex3}
    }
    \quad
    \subfloat[Die Regel $B \rightarrow Aa$ wird eingefügt. Dadurch kommt wird die Regel $A$ nur noch einmal vor, und die rechte Seite von $B$ ersetzt das eine Vorkommen von $A$ in der Datenstruktur.]{
        \begin{tikzpicture}[
            scale=.9, transform shape,
            normalnode/.style={rectangle, draw=black!60, thick, minimum size=5mm},
            guardnode/.style={diamond, draw=black!60, thick, minimum size=3mm},
            ]
            \tikzstyle{every node} = [circle, minimum size=7mm]
            
            % Rule S
            
            \node[normalnode] (S) at (0, 0) {S};
            \node[guardnode] (G1) at (1, 0) {};
            \node[normalnode] (B11) at (2, 0) {B};
            \node[normalnode] (c1) at (3, 0) {c};
            \node[normalnode] (B12) at (4, 0) {B};
            
            \draw [ -> ]  (S)  -- (G1);
            
            \foreach \from/\to in {G1/B11, B11/c1, c1/B12}
            \draw [ <-> ] (\from) -- (\to);
            
            \draw [ <-> ] (B12) to[out=90, in=90, looseness=0.5] (G1);
            
            % Rule B
            
            \node[normalnode] (B) at (2, -1.5) {B};
            \node[guardnode] (G3) at (3, -1.5) {};
            \node[normalnode, draw=blue!60] (a31) at (4, -1.5) {a};
            \node[normalnode, draw=blue!60] (b31) at (5, -1.5) {b};
            \node[normalnode] (a32) at (6, -1.5) {a};
            
            
            \draw [ -> ]  (B)  -- (G3);
            
            \foreach \from/\to in {G3/a31, a31/b31, b31/a32}
            \draw [ <-> ] (\from) -- (\to);
            
            \draw [ <-> ] (a32) to[out=90, in=45, looseness=1] (G3);
            
            % Connect S and B
            
            \draw [ -> ] (B11) to[out=-90, in=90] (G3);
            \draw [ -> ] (B12) to[out=-90, in=90] (G3);
            
        \end{tikzpicture}
        \label{seqex4}
    }

    \subfloat[Die Regel $C \rightarrow ab$ wurde erstellt. Der Algorithmus terminiert, da alle Zeichen aus dem Eingabetext eingelesen sind.]{
        \begin{tikzpicture}[
            scale=.9, transform shape,
            normalnode/.style={rectangle, draw=black!60, thick, minimum size=5mm},
            guardnode/.style={diamond, draw=black!60, thick, minimum size=3mm},
            ]
            \tikzstyle{every node} = [circle, minimum size=7mm]
            
            % Rule S
            
            \node[normalnode] (S) at (0, 0) {S};
            \node[guardnode] (G1) at (1, 0) {};
            \node[normalnode] (B11) at (2, 0) {B};
            \node[normalnode] (c1) at (3, 0) {c};
            \node[normalnode] (B12) at (4, 0) {B};
            \node[normalnode, draw=blue!60] (C11) at (5, 0) {C};
            
            \draw [ -> ]  (S)  -- (G1);
            
            \foreach \from/\to in {G1/B11, B11/c1, c1/B12, B12/C11}
            \draw [ <-> ] (\from) -- (\to);
            
            \draw [ <-> ] (C11) to[out=90, in=90, looseness=0.5] (G1);
            
            % Rule B
            
            \node[normalnode] (B) at (1, -1.5) {B};
            \node[guardnode] (G3) at (2, -1.5) {};
            \node[normalnode, draw=blue!60] (C31) at (3, -1.5) {C};
            \node[normalnode] (a31) at (4, -1.5) {a};
            
            
            \draw [ -> ]  (B)  -- (G3);
            
            \foreach \from/\to in {G3/C31, C31/a31}
            \draw [ <-> ] (\from) -- (\to);
            
            \draw [ <-> ] (a31) to[out=90, in=45, looseness=1] (G3);
            
            % Connect S and B
            
            \draw [ -> ] (B11) to[out=-90, in=90] (G3);
            \draw [ -> ] (B12) to[out=-90, in=90] (G3);
            
            
            % Rule C
            
            \node[normalnode] (C) at (3, -3) {C};
            \node[guardnode] (G2) at (4, -3) {};
            \node[normalnode] (a2) at (5, -3) {a};
            \node[normalnode] (b) at (6, -3) {b};
            
            \draw [ -> ]  (C)  -- (G2);
            
            \foreach \from/\to in {G2/a2, a2/b}
            \draw [ <-> ] (\from) -- (\to);
            
            \draw [ <-> ] (b) to[out=90, in=45, looseness=1] (G2);
            
            % Connect others to C
            
            \draw [ -> ] (C31) to[out=-90, in=90] (G2);
            \draw [ -> ] (C11) to[out=-90, in=90] (G2);
            
        \end{tikzpicture}
    }
    \caption{Der Aufruf von Sequitur auf den String $abacabab$}
    \label{sequiturexample}
\end{figure}

Lese nacheinander die Zeichen $a$, $b$, $a$, $c$, $a$ und $b$ ein. In jedem Schritt (außer im Ersten) wird das neu entstandene Digramm in die Hashtabelle eingefügt. Mit dem zuletzt gelesenen $b$ ist Digram-Uniqueness verletzt, da jetzt $ab$ doppelt in der Datenstruktur vorkommt, wie in \autoref{seqex2} zu sehen ist. Dies kann durch einen Lookup in der Hashtabelle festgestellt werden.

Wir erzeugen nun ein neues Nichtterminal $A$ und eine zugehörige Produktionsregel $A \rightarrow ab$  und ersetzen beide Vorkommen in der Grammatik durch das Nichtterminal. Das bereits vorher gelesene Vorkommen von $ab$ kann dabei mit der Hashtabelle gefunden werden. 
Die Nichtterminale enthalten eine Referenz auf die gemeinsam genutzte Instanz der Produktionsregel. 
An jeder Stelle, an der $ab$ durch $A$ ersetzt wurde, werden die nun nicht mehr existierenden Digramme aus der Hashtabelle entfernt, und die durch das Einsetzen von $A$ neu entstandenen Digramme in die Hashtabelle eingefügt.

Nun ist keine Eigenschaft mehr verletzt und wir lesen das Zeichen $a$. Digram-Uniqueness ist jetzt verletzt, da $Aa$ mehrmals vorkommt. (siehe \autoref{seqex3})
Wir verfahren wieder wie vorher um die Verletzung dieser Eigenschaft zu beheben und fügen die Regel $B \rightarrow Aa$ ein.

Die Verletzung von Digram-Uniqueness ist behoben, aber nun ist Rule-Utility verletzt, da die Produktionsregel von $A$ nur einmal (in der Regel $B$) verwendet wird.
Um dies zu beheben, wird das eine Vorkommen von $A$ gelöscht und die zugehörige Produktionsregel entfernt. Die rechte Seite von $A \rightarrow ab$ wird eingefügt, wo vorher das eine Vorkommen von $A$ war. Danach sieht die Grammatik wie in \autoref{seqex4} aus.

Damit sind alle Verletzungen behoben und wir lesen nacheinander die Zeichen $a$ und $b$.
Wieder ist Digram-Uniqueness verletzt, da $ab$ doppelt in der Grammatik vorkommt und wir erzeugen eine Regel $C \rightarrow ab$, um die Verletzung zu beheben. Nun sind alle Zeichen aus dem Eingabestring eingefügt und der Algorithmus terminiert. Die resultierende Grammatik ist also:

\begin{align*}
	S &\rightarrow BcBC\\
	B &\rightarrow Ca\\
	C &\rightarrow ab
\end{align*}


\tikzset{   
    arr/.style={
        matrix of nodes, 
        nodes={draw, minimum size=6mm, anchor=center}, 
        column sep=-\pgflinewidth,
        nodes in empty cells
    },
    queue/.style={
        matrix of nodes, 
        nodes={minimum size=6mm, anchor=center}, 
        row sep=3mm, 
        column sep=1cm, 
        nodes in empty cells
    }
}

\section{Re-Pair}
\label{repair}

Im Gegensatz zu Sequitur ist Re-Pair \cite{larsson_off-line_2000} ein Off-line Algorithmus. Es liegt also zur Ausführung bereits der gesamte Eingabestring im Speicher vor.

Ähnlich wie Sequitur ersetzt Re-Pair wiederholt vorkommende Paare von Zeichen durch Produktionsregeln, um den Text zu komprimieren. Allerdings wählt Re-Pair in jedem Schritt das am häufigsten vorkommende Symbolpaar und ersetzt jedes Vorkommen des Paars durch eine neue Regel.

\subsection{Implementation}

Re-Pair nutzt drei Datenstrukturen:

\subsubsection{Symbolarray}

Das \emph{Symbolarray} speichert alle Symbole der Startregel. Zu Anfang sind das genau die Zeichen des Eingabestrings. Jeder Eintrag im Array hat drei Teile. Einer ist das Symbol selbst, die anderen zwei sind Pointer auf die jeweils vorherige und folgende nicht-leere Position im Array. Diese dienen dazu, die einzelnen Symbole doppelt zu verketten, wenn Lücken im Array entstehen.
\begin{figure}
	\centering
	\begin{tikzpicture}
		
		\matrix (A) [arr]{
			a & b & c & a & b & c & d\\
		};
		
		\begin{scope}[yshift=-1.5cm]
			\matrix (B) [arr]{
				a & A &  & a & A & & d\\
			};
			
			\draw [<->] (B-1-2) to[out=90, in=90] (B-1-4);
			\draw [<->] (B-1-5) to[out=90, in=90] (B-1-7);
		\end{scope}
	\end{tikzpicture}
	\caption{Ein Beispiel für das Symbolarray für den String $abcabcd$ und die Ersetzung des Digramms $bc$. Abgebildet ist das Symbolarray, bevor und nachdem $bc$ durch $A$ ersetzt wurde. Die leeren Einträge werden durch Pointer überbrückt. Es existieren auch Pointer zwischen Einträgen, falls keine leeren Einträge dazwischen liegen. Diese sind hier aus Gründen der Leserlichkeit nicht abgebildet.}
    \label{symbarr}
\end{figure}
\autoref{symbarr} zeigt ein Beispiel für den String $abcabcd$ und die Ersetzung des Paars $bc$.

\subsubsection{Prioritätswarteschlange}

Re-Pair benutzt eine spezialisierte Prioritätswarteschlange, um die Wahl des günstigsten zu ersetzenden Paars effizient zu ermöglichen. Diese besteht aus einem Array von etwa $\sqrt{n}$ verketteten Listen, wobei $n$ die Länge der Eingabe ist. In der Liste an Index $i$ befinden sich alle Paare, die $i$-mal im String vorkommen. Zusätzlich enthalten die Einträge auch  einen Pointer auf das erste Vorkommen des Paares im String.

Eine Besonderheit ist dabei die Liste bei Index $0$. Diese enthält alle Paare, die öfter als $\sqrt{n}$-mal vorkommen. Außerdem werden keine Elemente in der Liste mit Index $1$ gespeichert, da diese nie zum Ersetzen in Betracht bezogen werden können.
 
\subsubsection{Symbolpaar-Hashtabelle}

Dies ist eine Hashtabelle, die einen Eintrag für jedes aktive Symbolpaar besitzt. Ein Symbolpaar heißt aktiv, wenn dieses Paar noch potenziell ersetzt werden kann, es also öfter als nur einmal vorkommt. Jeder Eintrag enthält einen Pointer auf den entsprechenden Eintrag in der Prioritätswarteschlange für dieses Symbolpaar.
\begin{figure}
	\centering
    \subfloat[Die Datenstrukturen nach der Initialisierung.]{
        \begin{tikzpicture}
		
            \node[draw=none] (Symbols) at (0,0) {Symbolarray};
            \node[draw=none, below=1.5cm of Symbols] (Queue) {Warteschlange};
            
            \matrix (A) [arr, right=1cm of Symbols, ampersand replacement=\&]{
                a \& b \& c \& a \& b \& c \& d\\
            };
            
            \matrix (lists) [queue, right=1cm of Queue, ampersand replacement=\&]{
                \& |[draw]| bc \& \\
                \& |[draw]| ab \& \\
                $>3$ \& $2$ \& $3$ \\
            };
            
            \draw [->] (lists-1-2) to[out=180, in=-90] (A-1-2);
            \draw [->] (lists-2-2) to[out=180, in=-90] (A-1-1);
            \draw [<->] (lists-1-2) -- (lists-2-2);
            
            
            \matrix (hash) [arr, right=1cm of lists-1-3]{
                bc\\
                ab\\
            };
        
            \node[draw=none, rotate=90, right=5mm of hash, anchor=center] (Hashtabelle) {Hashtabelle};
            
            \draw [->] (hash-1-1) to[out=180, in=0] (lists-1-2);
            \draw [->] (hash-2-1) to[out=180, in=0] (lists-2-2);
            
        \end{tikzpicture}
        \label{repairex1}
    }
	
    \subfloat[$ab$ wurde durch $A$ ersetzt. Die Einträge in der Hashtabelle und der Warteschlange wurden aktualisiert und nun kommt nur noch das Digramm $Ac$ mehrmals vor.]{
        \begin{tikzpicture}
		
            \node[draw=none] (Symbols) at (0,0) {Symbolarray};
            \node[draw=none, below=1cm of Symbols] (Queue) {Warteschlange};
            
            \matrix (A) [arr, right=1cm of Symbols, ampersand replacement=\&]{
                A \& \& c \& A \& \& c \& d\\
            };
        
            \draw[<->] (A-1-1) to[in=90, out=90] (A-1-3);
            \draw[<->] (A-1-4) to[in=90, out=90] (A-1-6);
        
            
            \matrix (lists) [queue, right=1cm of Queue, ampersand replacement=\&]{
                \& |[draw]| Ac \& \\
                $>3$ \& $2$ \& $3$ \\
            };
            
            \draw [->] (lists-1-2) to[out=180, in=-90] (A-1-1);
            
            
            \matrix (hash) [arr, right=1cm of lists-1-3]{
                Ac\\
            };
            
            
            \node[draw=none, rotate=90, right=5mm of hash, anchor=center] (Hashtabelle) {Hashtabelle};
            
            \draw [->] (hash-1-1) to[out=180, in=0] (lists-1-2);
            
        \end{tikzpicture}
        \label{repairex2}
    }

    \subfloat[$Ac$ wurde durch $B$ ersetzt. Da die Warteschlange leer ist, terminiert der Algorithmus.]{
        \begin{tikzpicture}
		
            \node[draw=none] (Symbols) at (0,0) {Symbolarray};
            \node[draw=none, below=0.5cm of Symbols] (Queue) {Warteschlange};
            
            \matrix (A) [arr, right=1cm of Symbols, ampersand replacement=\&]{
                B \& \& \& B \& \& \& d\\
            };
            
            \draw[<->] (A-1-1) to[in=110, out=90] (A-1-4);
            \draw[<->] (A-1-4) to[in=90, out=70] (A-1-7);
            
            
            \matrix (lists) [queue, right=1cm of Queue, ampersand replacement=\&]{
                \& \& \\
                $>3$ \& $2$ \& $3$ \\
            };
            
            
        \end{tikzpicture}
        \label{repairex3}
    }
	\caption{Die Datenstrukturen für den String $abcabcd$ während des Aufrufs von RePair.}
    \label{repairexample}
\end{figure}
\autoref{repairexample} zeigt die Datenstruktur zur Zeit der Initialisierung für den String $abcabcd$. 

\subsection{Beispiel}

Als Beispiel wenden wir nun den Re-Pair Algorithmus auf $abcabcd$ an. Der Aufruf ist in \autoref{repairexample} abgebildet. Die Datenstrukturen zu Anfang sind in \autoref{repairex1} zu sehen.
Wir entnehmen ein Symbolpaar aus der Prioritätswarteschlange, das am häufigsten vorkommt und ersetzen es im Text. In unserem Fall ist das zum Beispiel $ab$. Die Datenstrukturen nach dieser Ersetzung sind in \autoref{repairex2} abgebildet.

Nun ist das einzige übrige Symbolpaar, das neu entstandene $Ac$. Dies entnehmen wir aus der Prioritätswarteschlange und ersetzen es ebenfalls.
Nach der Ersetzung von $Ac$ durch $B$ ist die Prioritätswarteschlange nun leer und damit terminiert der Algorithmus. (siehe \autoref{repairex3}) Die resultierende Grammatik ist:
\begin{align*}
	S &\rightarrow BBa\\
	A &\rightarrow ab\\
	B &\rightarrow Ac
\end{align*}