\chapter{Theoretischer Hintergrund}

\section{Notation}

Die hier aufgeführte Notation orientiert sich zum Großteil an der Notation in \cite{kieffer_grammar-based_2000}:
\begin{itemize}
	\item $|\Sigma|$ beschreibt für eine endliche Menge $\Sigma$ die Anzahl der Elemente in $\Sigma$
	\item $|x|$ beschreibt für einen endlichen String $x$ die Länge von $x$
	\item $\Sigma^*$ bezeichnet die Menge aller Strings, deren einzelne Symbole Elemente aus $\Sigma$ sind. Also gilt für $x_0, \dots, x_{n-1} \in \Sigma$, dass $x_0 x_1 \dots x_{n-1} \in \Sigma^*$. Außerdem liegt auch der leere String $\varepsilon$ in $\Sigma^*$
	\item Für Strings $x, y \in \Sigma^*$ mit $|x| = n$ und $|y| = k$ ist $xy \in \Sigma^*$ definiert als die Konkatenation von $x$ und $y$, also $xy = x_0 \dots x_{n-1} y_1 \dots y_{k-1}$ 
	\item Sei $x = x_0 x_1 \dots x_{n-1} \in \Sigma^*$ ein String. Für $i \leq j$ ist dann $x[i..j]$ definiert als $x_i x_{i+1} \dots x_j \in \Sigma^*$ und heißt Substring von $x$. Ein Substring $P_i := x[0..i]$ heißt für $0 \leq i < n$ $i$-ter Präfix von $x$ und ein Substring $S_i := x[i..n - 1]$ heißt für $0 \leq i < n$ $i$-tes Suffix von $x$.
\end{itemize}

\section{Verlustfreie Datenkompression}

Datenkompression ist die Kodierung von Informationen in einer Repräsentation, die weniger Bits als die Eingabe benötigt \cite{mahdi_implementing_2012}. 
Im Gegensatz zur verlustbehafteten Kompression, die Daten komprimiert, indem nur geringfügig wichtige Information aus der Eingabe entfernt wird, lässt sich bei verlustfreier Kompression die gesamte Eingabe aus der komprimierten Repräsentation wiederherstellen. 

Dabei ist die Kompressionsrate definiert als das Verhältnis von der Länge des komprimierten Strings zur Länge des Eingabestrings.

\section{Straight-Line-Grammatiken}

Eine kontextfreie Grammatik $G = (N, \Sigma, P, S)$ ist ein 4-Tupel mit Nichtterminalsymbolen $N$, Terminalsymbolen $\Sigma$, $N \cap \Sigma = \emptyset$ und einem Startsymbol $S \in N$.
Die Grammatik enthält Produktionsregeln $P$ der Form $A \rightarrow \alpha$, mit $A \in N$ und $\alpha \in (N \cup \Sigma)^*$.

Wir schreiben 
\begin{equation*}
	\alpha B \gamma \Rightarrow \alpha \beta \gamma
\end{equation*}
für $\alpha, \beta, \gamma \in (N \cup \Sigma)^*$ und $B \in N$, falls eine Produktion $B \rightarrow \beta$ in $P$ existiert. 

Für $\alpha_i \in (N \cup \Sigma)^*, i = 0 \dots n, n \geq 1$ schreiben wir dann 
\begin{equation*}
	\alpha_0 \xRightarrow{n} \alpha_n
\end{equation*}
falls $\alpha_0 \Rightarrow \alpha_1, \alpha_1 \Rightarrow \alpha_2, \dots, \alpha_{n-1} \Rightarrow \alpha_n$ gelten. Das heißt also, dass sich $a_n$ aus $a_0$ durch Anwenden von $n$ Produktionsregeln aus $P$ ableiten lässt.
Falls es zu $\alpha, \beta \in (N \cup T)^*$ ein $n \in \mathbb{N}$ gibt, sodass $\alpha \xRightarrow{n} \beta$ gilt, dann schreiben wir 
\begin{equation*}
	\alpha \xRightarrow{*} \beta
\end{equation*}
Die Sprache von $G$ ist dann definiert als $L(G) := \{x \in \Sigma^*\ |\ S \xRightarrow{*} x\}$.

$G$ heißt nun Straight-Line-Grammatik, falls $|L(G)| = 1$, also die Grammatik genau ein Wort erzeugt. Die Grammatik hat somit weder Verzweigungen noch Schleifen \cite{benz_effective_2013}. 

\section{Grammatikbasierte Kompression}

Das Ziel der grammatikbasierten Kompression ist es, zu einem Eingabestring 
\begin{equation*}
	x = x_0 \dots x_{n-1} \in \Sigma^*
\end{equation*}
für ein Alphabet $\Sigma$, eine Straight-Line-Grammatik $G$ zu erzeugen, für die $L(G) = \{x\}$ gilt. Dabei sollte $G$ möglichst klein sein. 

Dazu in Verbindung steht das Problem, die kleinste Straight-Line-Grammatik zu einem String zu finden \cite{charikar_smallest_2005}. 
Dieses Problem ist allerdings $\mathcal{NP}$-vollständig \cite{charikar_smallest_2005}. Es ist nun also die Aufgabe, eine möglichst gute Approximation zu finden.
Die Größe einer Grammatik sei dabei definiert als die Summe der Länge aller rechten Seiten von Produktionsregeln in $P$ \cite{charikar_smallest_2005}, also:

\begin{equation*}
	|G| := \sum_{(A \rightarrow \alpha) \in P} |\alpha|
\end{equation*}

Die grundlegende Idee ist, für wiederholte Vorkommen von Substrings $x[i..j]$ in $x$ eine Produktion $A \rightarrow x[i..j]$ zu erzeugen und alle Vorkommen dieses Substrings durch das neue Nichtterminal $A \notin \Sigma$ zu ersetzen.

Es gibt Online-Algorithmen \cite{dinklage_compression_2017}, die die Eingabe von links nach rechts einlesen und schrittweise während des Lesens eine Grammatik aufbauen und anpassen. Ein Vertreter von Online-Algorithmen ist etwa Sequitur \cite{nevill-manning_identifying_1997}. 
Andererseits gibt es Offline-Algorithmen \cite{dinklage_compression_2017}, die die gesamte Eingabe in den Speicher laden und ganz betrachten um zu ersetzende Substrings auszuwählen. Ein Beispiel wäre Re-Pair \cite{larsson_offline_1999}.

Mit Methoden, eine möglichst kleine Grammatik zu erzeugen, wurde sich bereits beschäftigt \cite{benz_effective_2013, carrascosa_choosing_2010, charikar_smallest_2005}. 

\section{Enhanced Suffix-Array}

Das Enhanced Suffix-Array \cite{abouelhoda_replacing_2004} beschreibt im Allgemeinen das Suffix-Array \cite{manber_suffix_1993}, das mit verschiedenen anderen Tabellen, die im Folgenden beschrieben werden, ausgestattet wurde.

Es sei nun $x = x_0 x_1 \dots x_{n-1}\in \Sigma^*$ ein String mit $|x| = n$ über dem Alphabet $\Sigma$ 

\subsection{Suffix-Array}

Das Suffix-Array ist ein Array $SA$ mit $n$ Indizes $0 \leq i < n$, so dass 
\begin{equation*}
	S_{SA[0]}, S_{SA[1]}, \dots, S_{SA[n-1]}
\end{equation*}
die lexikographisch sortierte Reihenfolge der Suffixe des String $x$ ist \cite{abouelhoda_replacing_2004, manber_suffix_1993}.

Ein Eintrag $SA[i]$ beschreibt also den Index in $x$, an dem das lexikographisch $i$-kleinste (bei $0$ anfangend) Suffix beginnt. $SA$ kann in Laufzeit $\mathcal{O}(n)$ berechnet werden \cite{nong_two_2011}. 

Oftmals (etwa \cite{fischer_inducing_2011, nong_two_2011}) wird an $x$ noch ein weiteres Symbol $\$ \notin \Sigma$ angehängt, das lexikographisch kleiner als alle Zeichen in $x$ ist. Für den String $x\$ = x_0\dots x_{n-1} \$$ und das zugehörige Suffix-Array gilt dann: $SA[0] = n$ und $S_{SA[0]} = \$$. Dieses Suffix repräsentiert dann das leere Suffix.

\subsection{Inverses Suffix-Array}

Das Inverse Suffix-Array zu einem Suffix-Array $SA$ ist ein Array $ISA$ mit $n$ Indizes \\
$0 \leq i < n$ so dass $SA[ISA[i]] = i$.
Ein Eintrag $ISA[i]$ beschreibt also den Suffix-Array Index des Suffixes $S_i$ von $x$.

$ISA$ kann trivial in Laufzeit $\mathcal{O}(n)$ nach der obigen Vorschrift berechnet werden \cite{abouelhoda_replacing_2004}.

\subsection{LCP-Array}

Das LCP-Array ist ein Array $LCP$, für das zu einem gegebenen Suffix-Array $SA$ gilt: 

Es sei $LCP[0] := 0$. Es gilt $LCP[i] = k$ für $1 \leq i < n$ wenn der längste gemeinsame Präfix von $S_{SA[i-1]}$ und $S_{SA[i]}$ die Länge $k$ hat. $LCP[i]$ ist also die Länge des längsten gemeinsamen Präfixes zweier lexikographisch benachbarter Suffixe von $x$. 

Das LCP-Array kann mithilfe von $SA$ \cite{kasai_linear-time_2001} oder alternativ während der Berechnung von SA \cite{fischer_inducing_2011} in $\mathcal{O}(n)$ Laufzeit berechnet werden.